\documentclass[a4paper]{article}
\usepackage{impostazioni-prob}

% TODO: controllare forma pdf, poi controllare forma sorgente
% TODO: aggiungere titoli teoremi/definizioni
% TODO: aggiungere numerazione formule
% TODO: controllare spacing sezioni
% TODO: riflessione: in P(blabla) blabla non dovrebbe essere sempre un insieme? e in P(X=x) e simili?
% TODO: \qedhere
% TODO: assiomi della probabilità continua?
\begin{document}
\title{Probabilità}
\author{Alessandro Clerici Lorenzini}
\date{anno accademico 2020/21}
\maketitle
\tableofcontents

\subsection*{Notazione}
Alcune considerazioni sulle convenzioni di notazione scelte:
\begin{itemize}
	\item per il valore atteso di $X$ si usa $\ev{X}$, e non $\mathcal{E}(X)$.
	\item $P(A,B):=P(A\land B)$ o talvolta $P(A,B):=P(A\cap B)$
\end{itemize}


\section{Definizioni}
La definizione assiomatica della probabilità riassume in pochi punti quei concetti che si possono definire intuitivi nel calcolo delle probabilità.



\subsection{L'algebra degli eventi}
\begin{defin}
	Scelto un insieme $\Omega$ detto spazio campionario o degli eventi, si dice esito un elemento $\omega\in\Omega$ dell'insieme ed evento un suo sottoinsieme $E\subseteq\Omega$.

\end{defin}
\begin{defin}
	Sia $\A\in2^\Omega$ una collezione di sottoinsiemi di $\Omega$. Allora $\A$ è un'algebra se
	\begin{align*}
		 & \bullet ~ \Omega\in \A                                                                                            \\
		 & \bullet ~ \forall E\subseteq\Omega\qquad E\in\A\Rightarrow \bar E\in\A \qquad\text{con }\bar E:=\Omega\setminus E \\
		 & \bullet ~ \forall E_1,\dots E_n\in\Omega\qquad \forall i~E_i\in\A\Rightarrow \bigcup\limits_{i=1}^n E_i \in\A
	\end{align*}
	$\A$ è una $\sigma$-algebra su $\Omega$ se l'ultima condizione si può estendere a unioni numerabili qualsiasi.
\end{defin}




\subsection{Assiomi di Kolmogorov}
La probabilità viene definita come una funzione di un'algebra degli eventi $\A$ in $\R$:
\begin{equation*}
	P:\A\to\R
\end{equation*}

I seguenti assiomi, detti di Kolmogorov, decretano le proprietà che la probabilità rispetta:


\subsubsection{Primo assioma}
L'immagine di $P$ è l'insieme $[0,1]\in\R$. Equivalentemente, la probabilità di qualunque evento è compresa tra $0$ e $1$:
\begin{equation*}
	\forall E\in\A\qquad 0\leq P(E)\leq 1
\end{equation*}


\subsubsection{Secondo assioma}
La probabilità dello spazio campionario è $1$:
\begin{equation*}
	P(\Omega)=1
\end{equation*}


\subsubsection{Terzo assioma}
La probabilità dell'unione di eventi mutuamente esclusivi, cioè disgiunti (intuitivamente, il cui avvenire dell'uno esclude l'avvenire dell'altro), è uguale alla somma delle probabilità dei singoli:
\begin{equation*}
	\forall E_1,\dots,E_n\in\A\qquad \forall i,j~E_i\cap E_j = \emptyset \Rightarrow P\left( \bigcup_{i=1}^n E_i \right)=\sum_{i=1}^n P(E_i)
\end{equation*}



\subsection{Teoremi elementari}
Dagli assiomi di Kolmogorov derivano alcune proprietà elementari facilmente dimostrabili.


\subsubsection{Probabilità dell'evento complementare}
\begin{defin}
	Dato un evento $E\in\A$, l'evento complementare è l'evento $\bar E := \Omega\setminus E$.
\end{defin}
\begin{teor}[probabilità dell'evento complementare] \label{t:probcompl}
	Dato un evento $E$, se la probabilità di $E$ è $P(E)$, la probabilità dell'evento complementare di $E$ è $1-P(E)$:
	\begin{equation*}
		\forall E\in\A\qquad P(\bar E)=1-P(E)
	\end{equation*}
\end{teor}
\begin{proof}
	\begin{align*}
		  & \left.
		\begin{array}{cc}
			E\cap\bar E=\emptyset \\
			E\cup\bar E=\Omega
		\end{array} \right\}  \bc{definizione di evento complementare} \\
		1 & = P(\Omega)        \bc{secondo assioma}                                  \\
		  & = P(E\cup\bar E)                                                         \\
		  & = P(E)+P(\bar E)   \bc{terzo assioma}
	\end{align*}
	ergo:
	\begin{equation*}
		P(\bar E)=1-P(E) \qedhere
	\end{equation*}
\end{proof}

\subsubsection{Probabilità dell'unione}
\begin{teor} \label{t:probunion}
	Dati due eventi $E,F\in\A$, la probabilità della loro unione è uguale alla somma delle loro probabilità meno la probabilità dell'intersezione:
	\begin{equation*}
		\forall E,F \in\A\qquad P(E\cup F)=P(E)+P(F)-P(E\cap F)
	\end{equation*}
\end{teor}
\begin{proof}
	L'unione degli eventi è scrivibile come l'unione di due insiemi disgiunti:
	\begin{equation*}
		E\cup F=E \cup (\bar E\cap F) \\[1ex]
	\end{equation*}

	Passando alle probabilità:
	\begin{align*}
		P(E\cup F) & = P(E)+P(\bar E\cap F)                                               \\
		           & = P(E)+P(\bar E\cap F)+P(E\cap F)-P(E\cap F)                         \\
		           & = P(E)+P((\bar E\cap F)\cup (E\cap F))-P(E\cap F) \bc{terzo assioma} \\
		           & = P(E)+P(F)-P(E\cap F)                            \qedbc
	\end{align*}
\end{proof}

\subsubsection{Probabilità dell'evento vuoto}
La probabilità dell'evento vuoto ($\emptyset$) è $0$. Un evento con probabilità nulla viene detto evento impossibile.
\begin{equation*}
	P(\emptyset)=0
\end{equation*}

\begin{proof}
	\begin{align*}
		P(\Omega)    & = 1 \bc{secondo assioma}                       \\
		P(\emptyset) & = P(\bar\Omega)                                \\
		             & = 1 - P(\Omega) \bc{teorema \ref{t:probcompl}} \\
		             & = 1 - 1 = 0     \qedbc
	\end{align*}
\end{proof}



\subsection{Spazi di probabilità}
\begin{defin}
	Uno spazio di probabilità è una tripla $(\Omega,\A,P)$ composta da uno spazio campionario $\Omega$, un'algebra degli eventi $\A$ e una funzione di probabilità $P$.
\end{defin}


\subsubsection{Spazi equiprobabili}
\begin{defin}
	Uno spazio è equiprobabile se gli eventi elementari (cioè corrispondenti a singoletti) hanno probabilità costante $p$.
\end{defin}

Un evento elementare è composto da un singolo esito, cioè è un singoletto dell'insieme delle parti dello spazio campionario.
\begin{teor}
	In uno spazio equiprobabile, la probabilità di ogni evento elementare $e$ è uguale al reciproco del numero $n$ degli eventi elementari (che sono ovviamente a due a due disgiunti):
	\begin{equation}
		P(e)=\frac{1}{n}
	\end{equation}
\end{teor}
\begin{proof}
	\begin{align*}
		1=P(\Omega)=P\left( \bigcup_{i=1}^n e_i \right) \bc{secondo assioma} \\
		=\sum_{i=1}^n P(e_i) = np \bc{terzo assioma}                         \\
	\end{align*}
	Ergo:
	\begin{equation*}
		\forall e_i ~ P(e_i)=p=\frac{1}{n} \qedhere
	\end{equation*}
\end{proof}

Gli eventi non elementari possono essere espressi come unione di eventi elementari.
\begin{teor}
	In uno spazio equiprobabile, dato un evento $E=\{e_1,\dots,e_k\}$:
	\begin{equation}
		P(E)=\frac{|E|}{n}
	\end{equation}
\end{teor}
\begin{proof}
	\begin{equation*}
		P(E)= \sum_{i=1}^{|E|} P(e_i) = \frac{|E|}{n} \qedhere
	\end{equation*}
\end{proof}

\section{Probabilità condizionata}
Il concetto di probabilità condizionata restringe lo spazio campionario all'evento condizionante.

\begin{defin}
	La probabilità condizionata da un evento $F$ (evento condizionante) su un evento $E$ (evento condizionato) viene definita, quando $F$ non è impossibile (ossia per $P(F)\neq 0$), come segue:
	\begin{equation*}
		P(E\mid F)=\frac{P(E \cap F)}{P(F)}
	\end{equation*}
\end{defin}

Dalla definizione di probabilità condizionata deriva la \textbf{regola di fattorizzazione}:
\begin{equation}
	P(E \cap F)=P(E\mid F)\cdot P(F)
\end{equation}



\subsection{Teorema delle probabilità totali}

\subsubsection{Caso particolare}
Dati eventi $E, F\in \A$, l'evento $E$ può essere scritto come la parte di $E$ che non contiene elementi di $F$, ossia $E\cap \bar F$, unita disgiuntamente a $E \cap F$:
\begin{align*}
	(E \cap \bar F) \cap (E \cap F) & = E \cap (F \cap \bar F)\bc{proprietà distributiva}  \\
	                                & = E \cap \emptyset = \emptyset                       \\[2ex]
	(E \cap \bar F) \cup (E \cap F) & = E \cap (F \cup \bar F) \bc{proprietà distributiva} \\
	                                & = E \cap \Omega = E
\end{align*}

Avendo espresso $E$ come unione di eventi disgiunti, si può applicare il terzo assioma:
\begin{equation*}
	P(E) = P((E \cap \bar F)\cup(E \cap F)) = P(E \cap \bar F)+P(E \cap F)
\end{equation*}
e, per la regola di fattorizzazione, assunto $P(F)\neq 0 \land P(\bar F)\neq 0$:
\begin{equation*}
	=P(E\mid \bar F)\cdot P(\bar F) + P(E \mid F)\cdot P(F)
\end{equation*}

\subsubsection{Forma generale}
\begin{teor}
	Data una partizione di $\Omega$ composta da eventi non impossibili $F_1,\dots, F_n$ e un evento $E$:
	\begin{equation*}
		P(E)= \sum_{i=1}^n P(E \mid F_i)\cdot P(F_i)
	\end{equation*}
\end{teor}
\begin{proof}
	La dimostrazione avviene in analogia con il caso particolare. Questa volta, tuttavia, gli eventi della partizione sono disgiunti per ipotesi e non in quanto complementari:
	\begin{equation*}
		P(E)=P \left(\bigcup_{i=1}^n (E\cap F_i) \right) = \sum_{i=1}^n P(E\cap F_i) = \sum_{i=1}^n P(E \mid F_i)\cdot P(F_i)
	\end{equation*}
\end{proof}


\subsection{Teorema di Bayes}
Nelle stesse ipotesi del teorema delle probabilità totali si dimostra un altro teorema, detto di Bayes:
\begin{teor}[di Bayes]
	Data una partizione di $\Omega$ composta da eventi non impossibili $F_1,\dots, F_n$ e un evento $E$:
	\begin{equation*}
		P(F_j\mid E)=\frac{P(E\mid F_j)\cdot P(F_j)}{\sum\limits_{i=1}^n P(E \mid F_i)\cdot P(F_i)}
	\end{equation*}
\end{teor}

\begin{proof}
	\begin{equation*}
		P(F_j\mid E)=\frac{P(F_j \cap E)}{P(E)}=\frac{P(E\mid F_j)\cdot P(F_j)}{\sum\limits_{i=1}^n P(E \mid F_i)\cdot P(F_i)}
	\end{equation*}
\end{proof}



\subsection{Eventi indipendenti}
\begin{defin}
	Eventi $E$ e $F$, con $P(F)>0$, si dicono indipendenti se $P(E \mid F)=P(E)$
\end{defin}

Equivalentemente, per definizione di probabilità condizionata
\begin{gather*}
	\frac{P(E \cap F)}{P(F)}=P(E) \\[1ex]
	P(E \cap F)=P(E)\cdot P(F)
\end{gather*}
e ovviamente, se $P(E)>0$:
\begin{equation*}
	P(F \mid E)=P(F)
\end{equation*}

\subsubsection{Proprietà}
\begin{teor}
	Se $E$ e $F$ sono indipendenti, allora $E$ e $\bar F$ sono indipendenti.
\end{teor}
\begin{proof}
	\begin{equation*}
		E = (E\cap F) \cup (E \cap \bar F)
	\end{equation*}
	\begin{align*}
		P(E \cap \bar F) & = P(E) - P(E \cap F)    \bc{terzo assioma}                          \\
		                 & = P(E) - P(E)\cdot P(F) \bc{in quanto $E$ ed $F$ sono indipendenti} \\
		                 & = P(E)(1-P(F))          \bc{teorema \ref{t:probcompl}}              \\
		                 & = P(E)\cdot P(\bar F)
	\end{align*}
\end{proof}

\subsubsection{Tripla di eventi indipendenti}
\begin{defin}
	Tre eventi $E$, $F$, $G$ sono indipendenti se
	\begin{itemize}
		\item $P(E \cap F)=P(E)\cdot P(F)$
		\item $P(E \cap G)=P(E)\cdot P(G)$
		\item $P(F \cap G)=P(F)\cdot P(G)$
		\item $P(E \cap F \cap G)=P(E)\cdot P(F)\cdot P(G)$
	\end{itemize}
\end{defin}

\noindent
Per quanto riguarda $E$, $F\cup G$:
\begin{align*}
	P(E \cap & (F\cup G))  = P((E\cap F)\cup (E\cap G))                \bc{distributiva}              \\
	         & = P(E \cap F) + P(E\cap G) - P((E\cap F)\cap (E\cap G)) \bc{teorema \ref{t:probunion}} \\
	         & = P(E)P(F) + P(E)P(G) - P(E)P(F)P(G)                    \bc{indipendenza}              \\
	         & = P(E)(P(F) + P(G)-P(F)P(G))
\end{align*}

\subsubsection{\texorpdfstring{$n$}{n}-upla di eventi indipendenti}
\begin{defin}[$n$-upla di eventi indipendenti]
	Eventi $E_1,\dots,E_n$ si dicono indipendenti se e solo se comunque sceltone $r$ la probabilità della loro intersezione è uguale al prodotto della probabilità dei singoli.
	\begin{equation*}
		\forall r<n\in\mathbb{N}~\forall~1\leq \alpha_1 \leq\dots\leq\alpha_r\leq n \qquad P \left(\bigcap_{i=1}^r E_{\alpha_i} \right)=\prod_{i=1}^r P(E_{\alpha_i})
	\end{equation*}
\end{defin}

\begin{examp}
	Dato un sistema in serie di dimensione $n$:
	\begin{gather*}
		\forall i=1,\dots,n \qquad p_i=P(\text{l'}i\text{-esimo componente funziona}) \\[1ex]
		P(\text{il sistema funziona}) = P(\text{tutti i componenti funzionanno}) = \\
		P\left(\bigcap_{i=1}^n \text{l'}i\text{-esimo componente funziona}\right)
	\end{gather*}

	In un sistema in parallelo:
	\begin{gather*}
		P(\text{il sistema funziona}) = 1-P(\text{il sistema non funziona}) = \\[1ex]
		1-P(\text{tutti i componenti non funzionano}) = \\[1ex]
		1- P\left(\bigcap_{i=1}^n \text{l'}i\text{-esimo componente non funziona}\right) = \\[1ex]
		1-\prod_{i=1}^n P(\text{l'}i\text{-esimo componente non funziona}) =
		1-\prod_{i=1}^n 1-p_i =
	\end{gather*}
\end{examp}

\section{Variabili aleatorie discrete}
\begin{defin}
	In uno spazio di probabilità $(\Omega, \A, p)$, una variabile aleatoria (o casuale) è una funzione\footnote{in verità, la funzione deve soddisfare il criterio: $\{\omega \mid X(\omega)\leq r\}\in\A\quad\forall r\in\R$} $X:\Omega\to \R$.
	L'immagine di una variabile aleatoria si dice supporto e si indica con $D_X$. Ogni elemento del supporto, cioè l'immagine per $X$ di un elemento di $\Omega$, si dice specificazione. Una specificazione generica di $X$ si indica tipicamente con $x$.
\end{defin}


\subsection{Definizioni}
\begin{defin}
	Una variabile aleatoria è discreta se e solo se il suo supporto è numerabile.
\end{defin}

Per esempio, se $\Omega$ è l'insieme di esiti del lancio di due dadi, $X$ può essere definito come la somma degli esiti dei due dadi. Le immagini di $X$ non sono equiprobabili. Se per $X$ si sceglie il valore del primo dado, le immagini sono equiprobabili.

\begin{defin}[funzione indicatrice]
	Dato $A\subseteq D$ (dove $D$ è un qualunque insieme), la funzione indicatrice (o caratteristica) di $A$ è la funzione $I:D\to \{0,1\}$ che associa a un elemento di $D$ il numero $1$ se l'elemento appartiene ad $A$, o $0$ altrimenti:
	\begin{equation*}
		I_A(x) := \begin{cases}
			1 \qquad x\in A \\
			0 \qquad x\notin A
		\end{cases}
	\end{equation*}
\end{defin}

\begin{defin}[funzione indicatrice di un evento] \label{def:findiceven}
	La funzione indicatrice di un evento $A$ è la variabile aleatoria che ha specificazione $1$ se l'evento $A$ si verifica e $0$ se l'evento non si verifica.
\end{defin}



\subsection{Funzione di ripartizione}
\begin{defin}[funzione di ripartizione] \label{def:fripar}
	Data una variabile aleatoria $X$, la funzione di ripartizione $F_X:\R\to[0,1]$ (o di distribuzione cumulativa) è la funzione che associa a un valore $x\in\R$ la probabilità che l'esito di $X$ ne sia minore o uguale:
	\begin{equation*}
		\forall x\in\R\quad F_X(x):=P(X\leq x)
	\end{equation*}
\end{defin}

% TODO: spiegare le formule seguenti
\begin{gather*}
	a\leq b \\
	x\sim F_X \\[1ex]
	\{x\leq b\}=\{x\leq a\}\cup\{a<x\leq b\} \\[1ex]
	P(x\leq b)=P(x\leq a)+P(a<x\leq b) \\[1ex]
	F_X(b)=F_X(a)+P(a<x\leq b) \\[1ex]
	P(a<x\leq b)=F_X(b)-F_X(a)
\end{gather*}

La funzione di ripartizione si annulla quandunque l'input è minore della minima specificazione.




\subsection{Funzione di massa di probabilità}
\begin{defin}[funzione di massa di probabilità] \label{def:fmassaprob}
	Data una variabile aleatoria discreta $X$, la funzione di massa di probabilità $p_X:\R\to[0,1]$ è la funzione che associa a un valore $x\in\R$ la probabilità che l'esito di $X$ sia uguale a $x$:
	\begin{equation*}
		\forall x \in\R\quad p_X(x)= P(x=X)
	\end{equation*}
\end{defin}

La somma delle immagini della funzione di massa di probabilità per tutte le specificazioni di $X$ è ovviamente uguale a $1$:
\begin{equation} \label{eq:sommamassa}
	\sum_i p_X (x_i) = 1
\end{equation}

La funzione di massa di probabilità di $X$ si annulla nei punti di $\R$ non appartenenti al supporto di $X$.

\begin{prop}
	Per le variabili aleatorie discrete vale la seguente relazione tra la funzione di ripartizione e la funzione di massa di probabilità:
	\begin{equation*}
		F_X(x)=\sum_{a\leq x} p_X(a)
	\end{equation*}
	con $a\in D_X$.
\end{prop}
\begin{proof}
	\begin{align*}
		F_X(x) & = P(X\leq x)                             \bc{definizione \ref{def:fripar}}     \\
		       & = P\left(\bigcup_{a\leq x}\{X=a\}\right)                                       \\
		       & = \sum_{a\leq x} P(X=a)                  \bc{unione di eventi disgiunti}       \\
		       & = \sum_{a\leq x} p_X(a)                  \bc{definizione \ref{def:fmassaprob}}
	\end{align*}
\end{proof}



\subsection{Valore atteso}
Il valore atteso di una variabile aleatoria $X$ è un indice di centralità delle specificazioni della variabile aleatoria.
\begin{defin}[valore atteso] \label{def:valatt}
	Data una variabile aleatoria discreta $X$, il valore atteso di $X$ è la media delle specificazioni di $X$ pesata con le rispettive probabilità:
	\begin{equation*}
		\ev{X}=\sum_i x_i P(X=x_i)
	\end{equation*}
	Talvolta si usa anche la notazione $\mu=\mu_X=\ev{X}$.
\end{defin}
\noindent
In generale, non è detto che la sommatoria che definisce il valore atteso converga.

\begin{prop} \label{prop:indvalatt}
	Il valore atteso della funzione indicatrice di un evento è uguale alla probabilità dell'evento:
	\begin{equation*}
		\ev{I_A}=P(A)
	\end{equation*}
\end{prop}
\begin{proof}
	Applicando la definizione \ref{def:valatt} di valore atteso e la \ref{def:findiceven} di funzione indicatrice di un evento:
	\begin{equation*}
		\ev{I_A}=1*P(A)+0*(1-P(A))=P(A)
	\end{equation*}
\end{proof}

\subsubsection{Linearità del valore atteso}
\begin{prop}
	Il valore atteso di una variabile aleatoria discreta $X$ è lineare rispetto a una trasformazione lineare $ax+b$ applicata alle specificazioni $x_i$ di $X$:
	\begin{equation*}
		D_Y = \{ax_i+b \mid x_i\in D_X\} \qquad\Rightarrow\qquad \ev{Y}=a\ev{X}+b
	\end{equation*}
	Si scrive anche $Y=aX+b$.
\end{prop}
\begin{proof}
	Poiché le probabilità per le rispettive specificazioni rimangono invariate:
	\begin{align*}
		\ev{Y} & = \sum_i (a x_i + b) P(X=x_i)                                                    \\
		       & = a\sum_i x_i P(X=x_i) + b\sum_i P(X=x_i)                                        \\
		       & = a\underbrace{\sum_i x_i p_X(x_i)}_{\ev{x}} + b\underbrace{\sum_i p_X(x_i)}_{1} \\
		       & = a \ev{x} + b \bc{\eqref{eq:sommamassa} e definizione \ref{def:valatt}}
	\end{align*}
\end{proof}



\subsection{Varianza di una variabile aleatoria}
\begin{defin}[varianza di una variabile aleatoria]
	La varianza di una variabile aleatoria $X$ è definita come segue:
	\begin{equation*}
		\var(X) = \sigma^2_X := \ev{(X-\ev{X})^2}
	\end{equation*}
\end{defin}

\begin{defin}[deviazione standard di una variabile aleatoria]
	La deviazione standard di una variabile aleatoria $X$ è così definita:
	\begin{equation*}
		\sigma(X)=\sqrt{\var(X)}
	\end{equation*}
\end{defin}


\subsubsection{Proprietà della varianza}
\begin{prop} \label{prop:varalt}
	La varianza di $X$ può essere espressa equivalentemente come la differenza tra il valore atteso del quadrato di $X$ e il valore atteso al quadrato di $X$:
	\begin{equation} \label{eq:varalt}
		\var(X) = \ev{x^2}- \ev{x}^2
	\end{equation}
\end{prop}
\begin{proof}
	\begin{align*}
		\var(X) & = \ev{X^2-2\mu X+\mu^2}         \bc{sviluppando il quadrato}     \\
		        & = \ev{X^2}-2\mu \ev{X}+\mu^2    \bc{linearità del valore atteso} \\
		        & = \ev{X^2}-2\mu^2+\mu^2         \bc{definizione di $\mu$}        \\
		        & = \ev{x^2}- \ev{x}^2
	\end{align*}
\end{proof}

\begin{prop}
	La varianza della funzione indicatrice di un evento $A$ è uguale al prodotto delle probabilità di $A$ e del suo complementare:
	\begin{equation*}
		\var(I_A) = P(A)P(\bar A)
	\end{equation*}
\end{prop}
\begin{proof}
	\begin{align*}
		\var(I_A) & = \ev{I_A^2}- \ev{I_A}^2                             \\
		          & = \ev{I_A}- \ev{I_A}^2   \bc{idempotenza di $I_A$}   \\
		          & = \ev{I_A}(1- \ev{I_A})  \bc{proprietà distributiva} \\
		          & = P(A)P(\bar A)
	\end{align*}
\end{proof}

\begin{prop}
	In una trasformazione lineare su una variabile aleatoria $X$, una traslazione non influisce sulla varianza; al contrario, una dilatazione si traduce in una dilatazione quadratica.
	\begin{equation*}
		\var(aX+b)=a^2\var(X)
	\end{equation*}
\end{prop}
\begin{proof}
	\begin{align*}
		\var(aX+b) & = \ev{(aX+b-\ev{aX+b})^2} \\
		           & = \ev{(aX+b-a\ev{X}-b)^2} \\
		           & = \ev{(a(X-\ev{X}))^2}    \\
		           & = a^2~\ev{(X-\ev{X})^2}   \\
		           & = a^2 \var(X)             \\
	\end{align*}
\end{proof}



\subsection{Funzione di ripartizione congiunta}
\begin{defin}
	Data una coppia di variabili aleatorie $X, Y$, la funzione di ripartizione congiunta $F_X:\R\times\R\to[0,1]$ (o di distribuzione cumulativa congiunta) è la funzione che associa a una coppia di valori reali per $x$ e $y$ la probabilità che l'esito di $X$ sia minore o uguale a $x$ e l'esito di $Y$ sia minore o uguale a $y$:
	\begin{equation*}
		\forall x,y\in\R\qquad F_{X,Y}(x,y):=P(X\leq x, Y\leq y) \\
	\end{equation*}
\end{defin}

Il limite per $y$ che tende a $+\infty$ della funzione di ripartizione congiunta di $X$ e $Y$ è uguale alla funzione di ripartizione di $X$. In questo caso si dice che $F_X(x)$ è la distribuzione marginale rispetto a $X$:
\begin{equation} \label{eq:distrmargin}
	\lim_{y\to+\infty}F_{X,Y}(x, y)=\lim_{y\to+\infty}P(X\leq x,Y\leq y)=\lim_{y\to+\infty}P(X\leq x)=F_X(x)
\end{equation}
Analogo vale, ovviamente, per $x\to+\infty$.



\subsection{Funzione di massa di probabilità congiunta}
\begin{defin}
	Data una coppia di variabili aleatorie $X, Y$, la funzione di massa di probabilità congiunta $p_{X,Y}:\R\times\R\to[0,1]$ è la funzione che associa a una coppia di valori reali per $x$ e $y$ la probabilità che l'esito di $X$ sia uguale a $x$ e quello di $Y$ sia uguale a $y$:
	\begin{equation*}
		p_{X,Y}(x,y)=P(X=x,Y=y)
	\end{equation*}
\end{defin}

Analogamente a quanto visto nella \eqref{eq:distrmargin} con il limite, la somma per tutte le specificazioni $y_j$ di $Y$ di $p_{X,Y}(x,y_j)$ è uguale alla massa di probabilità marginale rispetto a $X$:
\begin{equation} \label{eq:massamargin}
	\sum_j p_{X,Y}(x,y_j) = p_X(x)
\end{equation}




\subsection{Indipendenza}
\begin{defin}
	Variabili aleatorie $X$ e $Y$ sono indipendenti se e solo se
	\begin{equation*}
		\forall A,B\subseteq\R\qquad P(X\in A, Y\in B)=P(X\in A)P(Y\in B)
	\end{equation*}
\end{defin}

La definizione è equivalente alla seguente proprietà della funzione di massa di probabilità congiunta:
\begin{prop}
	Variabili aleatorie $X$ e $Y$ sono indipendenti se e solo se
	\begin{equation*}
		a,b\in\R\qquad p_{X,Y}(a,b)=p_X(a)p_Y(b)
	\end{equation*}
\end{prop}
\begin{proof}
	L'implicazione verso destra:
	\begin{gather*}
		\forall A,B\subseteq\R\qquad P(X\in A, Y\in B)=P(X\in A)P(Y\in B) \\
		\Rightarrow \\
		a,b\in\R\qquad p_{X,Y}(a,b)=p_X(a)p_Y(b)
	\end{gather*}
	è immediata dal momento che si possono scegliere sottoinsiemi di $\R$ corrispondenti a singoletti contenenti $a$ e $b$ ($A=\{a\}$ e $B=\{b\}$).

	L'implicazione inversa parte dalla ricostruzione di $A$ e $B$ come unioni di singoletti:
	\begin{align*}
		P(X\in A, Y\in B) & = \sum_{a\in A}\sum_{b\in B} p_{X,Y}(a,b)              \\
		                  & = \sum_{a\in A}\sum_{b\in B} p_X(a)p_Y(b) \bc{ipotesi} \\
		                  & = \sum_{a\in A}p_X(a) \sum_{b\in B} p_Y(b)             \\
		                  & = P(X\in A)P(Y\in B)
	\end{align*}
\end{proof}

Si dimostra anche la seguente equivalenza:
\begin{prop}
	Variabili aleatorie $X$ e $Y$ sono indipendenti se e solo se
	\begin{equation*}
		a,b\in\R\qquad F_{X,Y}(a,b)=F_X(a)F_Y(b)
	\end{equation*}
\end{prop}



\subsection{Multivariati}
I concetti visti per coppie di variabili aleatorie si estendono a vettori aleatori (più comunemente detti variabili aleatorie multivariate, o multivariati) di dimensione arbitraria:


\subsubsection{Funzione di ripartizione}
\begin{defin}
	Date $n$ variabili aleatorie $X_1, \dots, X_n$, la funzione di ripartizione $F_{X_1,\dots,X_n}:\R^n\to[0,1]$ è la funzione che associa a una $n$-upla di valori reali $x_1,\dots,x_n$ la probabilità che l'esito di $X_i$ sia minore o uguale a $x_i$ per ogni $i$ da $1$ a $n$:
	\begin{equation*}
		F_{X_1,\dots,X_n}(x_1,\dots,x_n)=P(X_1\leq x_1,\dots,X_n\leq x_n)
	\end{equation*}
\end{defin}


\subsubsection{Funzione di massa di probabilità}
\begin{defin}
	Date $n$ variabili aleatorie $X_1, \dots, X_n$, la funzione di ripartizione $F_{X_1,\dots,X_n}:\R^n\to[0,1]$ è la funzione che associa a una $n$-upla di valori reali $x_1,\dots,x_n$ la probabilità che l'esito di $X_i$ sia uguale a $x_i$ per ogni $i$ da $1$ a $n$:
	\begin{equation*}
		F_{X_1,\dots,X_n}(x_1,\dots,x_n)=P(X_1\leq x_1,\dots,X_n\leq x_n)
	\end{equation*}
\end{defin}


\subsubsection{Indipendenza}
\begin{defin}
	Variabili aleatorie $X_1,\dots,X_n$ sono indipendenti se e solo se
	\begin{equation*}
		\forall A_1,\dots,A_n\subseteq\R \qquad P\left(\bigcap_{i=1}^n X_i\in A_i\right) = \prod_{i=1}^n P(X_i\in A_i)
	\end{equation*}
\end{defin}


\subsubsection{Valore atteso}
Il valore atteso per una funzione applicata a una coppia di variabili aleatorie discrete coinvolge tutte le combinazioni\footnote{la sommatoria con doppio indice può essere scomposta in due sommatorie, una per ogni indice} di specificazioni delle due variabili:
\begin{equation*}
	\ev{f(x,y)}=\sum_{i,j}f(x_i,y_j)P(X=x_i,Y=y_i)
\end{equation*}

\begin{prop}
	Il valore atteso della somma di due variabili aleatorie discrete è uguale alla somma dei valori attesi delle stesse.
	\begin{equation*}
		\ev{X+Y}=\ev{X}+\ev{Y}
	\end{equation*}
\end{prop}
\begin{proof}
	\begin{align*}
		\ev{X+Y} & = \sum_i\sum_j (x_i+y_i) P(X=x_i,Y=y_i)                               \\
		         & = \sum_i\sum_j x_i P(X=x_i,Y=y_i) + \sum_i\sum_j y_i P(X=x_i,Y=y_i)   \\
		         & = \sum_i x_i \sum_j P(X=x_i,Y=y_i) + \sum_j y_i \sum_i P(X=x_i,Y=y_i)
	\end{align*}
	Le due sommatorie interne sono probabilità marginali:
	\begin{align*}
		 & = \sum_i x_i P(X=x_i) + \sum_j y_i P(Y=y_i) \\
		 & = \ev{X} + \ev{Y}
	\end{align*}
\end{proof}

Generalizzando:
\begin{prop}
	Il valore atteso della somma di variabili aleatorie discrete è uguale alla somma dei valori attesi delle stesse.
	\begin{equation*}
		\ev{\sum_i X_i}=\sum_i \ev{X_i}
	\end{equation*}
\end{prop}

Di conseguenza, se si può esprimere una variabile $X$ come la somma di variabili $X_1,\dots,X_n$, è sufficiente calcolare il valore atteso di ciascuna componente per calcolare quello della variabile iniziale.


\subsubsection{Altre proprietà}
Il seguente risultato è una generalizzazione della varianza, utile a valutare quanto un'approssimazione $c$ di $X$ è buona. La migliore approssimazione è $c=\ev{X}$, in coerenza con il senso semantico di $\ev{X}$.
\begin{prop}
	\begin{equation*}
		\ev{(X-c)^2}=\var(X)+(\mu-c)^2\geq\var(X)
	\end{equation*}
\end{prop}
\begin{proof}
	Se $\mu=\ev{X}$
	\begin{align*}
		\ev{(X-c)^2} & = \ev{(X-\mu+\mu-c)^2}                                                     \\
		             & = \ev{(X-\mu)^2-2(X-\mu)(\mu-c)+(\mu-c)^2}                                 \\
		             & = \ev{(X-\mu)^2}-\ev{2(X-\mu)(\mu-c)}+(\mu-c)^2                            \\
		             & = \ev{(X-\mu)^2}-2(\mu-c)\underbrace{\ev{X-\mu}}_{=\ev{X}-\mu=0}+(\mu-c)^2 \\
		             & = \ev{(X-\mu)^2}+(\mu-c)^2
	\end{align*}
\end{proof}



\subsection{Covarianza}

\begin{defin}
	Date due variabili aleatorie $X$ e $Y$, la covarianza tra $X$ e $Y$ è così definita:
	\begin{equation} \label{eq:covar}
		\cov(X,Y)=\ev{(X-\mu_X)(Y-\mu_Y)}
	\end{equation}
\end{defin}
\noindent
Ovviamente la covarianza è simmetrica, ossia $\cov(X,Y)=\cov(Y,X)$.


\subsubsection{Proprietà}
La seguente è una definizione equivalente per la covarianza tra due variabili aleatorie:
\begin{prop} \label{prop:covalt}
	Date due variabili aleatorie $X$ e $Y$, la covarianza tra $X$ e $Y$ è uguale al valore atteso del prodotto delle due meno il prodotto dei rispettivi valori attesi:
	\begin{equation*}
		\cov(X,Y)=\ev{XY}-\ev{X}\ev{Y}
	\end{equation*}
\end{prop}
\begin{proof}
	Sviluppando la \eqref{eq:covar}:
	\begin{align*}
		\cov(X,Y) & = \ev{XY-\mu_X Y-\mu_Y x + \mu_X\mu_Y}                                                            \\
		          & = \ev{X,Y}-\underbrace{\mu_X\ev{Y}}_{\mu_X\mu_Y}-\underbrace{\mu_Y\ev{X}}_{\mu_X\mu_Y}+\mu_X\mu_Y \\
		          & =\ev{XY}-\ev{X}\ev{Y}
	\end{align*}
\end{proof}

\begin{prop}
	La covarianza è lineare rispetto alla dilatazione di una variabile per una costante:
	\begin{equation*}
		\cov(aX,Y)=a\cov(X,Y)
	\end{equation*}
\end{prop}
\begin{proof}
	Applicando la proprietà \ref{prop:covalt}:
	\begin{align*}
		\cov(aX,Y) & = \ev{aXY}-\ev{aX}\ev{Y}  \\
		           & = a(\ev{XY}-\ev{X}\ev{Y}) \\
		           & = a\cov(X,Y)
	\end{align*}
\end{proof}

\begin{prop} \label{prop:covsum0}
	\begin{equation*}
		\cov(X+Y,Z)=\cov(X,Z)+\cov(Y,Z)
	\end{equation*}
\end{prop}
\begin{proof}
	\begin{align*}
		\cov(X+Y,Z) & = \ev{(X+Y)Z} - \ev{X+Y}\ev{Z}                    \\
		            & = \ev{XZ+YZ} - (\ev{X} + \ev{Y}) \ev{Z}           \\
		            & = \ev{XZ} + \ev{YZ} - \ev{X}\ev{Z} - \ev{Y}\ev{Z} \\
		            & = \cov(X,Z) + \cov(Y,Z)
	\end{align*}
\end{proof}

Generalizzando:
\begin{prop} \label{prop:covsum}
	\begin{equation*}
		\cov\left(\sum_i X_i,\sum_j Y_j\right)=\sum_i\sum_j\cov(X_i,Y_i)
	\end{equation*}
\end{prop}
\begin{proof}
	Per la :
	\begin{align*}
		\cov\left(\sum_i X_i,\sum_j Y_j\right) & = \sum_i\cov\left(X_i,\sum_j Y_j\right) \bc{proprietà \ref{prop:covsum0}} \\
		                                       & = \sum_i\sum_j \cov(X_i,Y_i) \bc{proprietà \ref{prop:covsum0}}            \\
	\end{align*}
\end{proof}

\begin{prop}
	La covarianza tra una variabile aleatoria $X$ e se stessa è la varianza di $X$:
	\begin{equation*}
		\cov(X,X)=\var(X)
	\end{equation*}
\end{prop}
\begin{proof}
	\begin{align*}
		\cov(X,X) & = \ev{(X-\mu)(X-\mu)} \\
		          & = \ev{(X-\mu)^2}      \\
		          & = \var(X)
	\end{align*}
\end{proof}

\begin{prop} \label{prop:varsum}
	La varianza della somma di due variabili aleatorie $X$ e $Y$ è uguale alla somma delle rispettive varianze più il doppio della loro covarianza:
	\begin{equation*}
		\var(X+Y) = \var(X)+\var(Y)+2\cov(X,Y)
	\end{equation*}
\end{prop}
\begin{proof}
	Usando la definizione alternativa (proprietà \ref{prop:varalt}):
	\begin{align*}
		\var(X+Y) & = \ev{(X+Y)^2} - \ev{X+Y}^2                                            \\
		          & = \ev{X^2+2XY+Y^2}-(\ev{X}+\ev{Y})^2                                   \\
		          & = \ev{X^2} + 2\ev{XY} + \ev{Y^2} - \ev{X}^2 - 2\ev{X}\ev{Y} - \ev{Y}^2 \\
		          & = \ev{X^2} - \ev{X}^2 + \ev{Y^2} - \ev{Y}^2 + 2(\ev{XY}-\ev{X}-\ev{Y}) \\
		          & = \var(X) + \var(Y) + 2\cov(X,Y)
	\end{align*}
\end{proof}

Generalizzando:
\begin{prop} \label{prop:varsumcov}
	\begin{equation*}
		\var\left(\sum_{i=1}^n X_i\right) = \sum_{i=1}^n \var(X_i) + \sum_{i\neq j}\cov(X_i,Y_j)
	\end{equation*}
\end{prop}
\noindent
La dimostrazione è ovvia

\begin{teor}
	Il valore atteso del prodotto di variabili aleatorie discrete indipendenti è uguale al prodotto dei rispettivi valori attesi:
	\begin{equation*}
		\ev{XY} = \ev{X}\ev{Y}
	\end{equation*}
\end{teor}
\begin{proof}
	\begin{align*}
		\ev{XY} & = \sum_i\sum_j x_i y_j P(X=x_i,Y=y_i)                              \\
		        & = \sum_i\sum_j x_i y_j P(X=x_i)P(Y=y_j)                            \\
		        & = \left(\sum_i x_i P(X=x_i)\right)\left(\sum_j y_j P(Y=y_j)\right) \\
		        & = \ev{X}\ev{Y}
	\end{align*}
\end{proof}

\begin{corol}
	La covarianza di ude variabili aleatorie discrete indipendenti $X$ e $Y$ è nulla:
	\begin{equation*}
		\cov(X,Y) = 0
	\end{equation*}
\end{corol}
\begin{proof}
	\begin{align*}
		\cov(X,Y) & = \ev{XY}-\ev{X}\ev{Y}      \\
		          & = \ev{X}\ev{Y}-\ev{X}\ev{Y} \\
		          & = 0
	\end{align*}
\end{proof}

\begin{corol}
	Date variabili indipendenti $X_1,\dots,X_n$:
	\begin{equation*}
		\var\left(\sum_{i=1}^n X_i\right) = \sum_{i=1}^n \var(X_i)
	\end{equation*}
\end{corol}
\begin{proof}
	Applicando la proprietà \ref{prop:varsum}:
	\begin{align*}
		\var\left(\sum_{i=1}^n X_i\right) & = \sum_{i=1}^n\var(X_i)+2\sum_{i\neq j}\cov(X_i,X_j) \\
		                                  & = \sum_{i=1}^n \var(X_i)
	\end{align*}
\end{proof}


\begin{prop}
	Date funzioni indicatrici di eventi, o in generale variabili aleatorie bernoulliane (ossia di specificazione binaria):
	\begin{equation*}
		\cov(X,Y)>0 \Rightarrow P(X=1\mid Y=1)>P(X=1)
	\end{equation*}
\end{prop}
\begin{proof}
	\begin{align*}
		\cov(X,Y) & = \ev{X,Y} - \ev{X}\ev{Y} \\
		          & = P(XY=1)-P(X=1)P(Y=1)    \\
		          & = P(X=1,Y=1)-P(X=1)P(Y=1) \\
	\end{align*}
	Se $\cov(X,Y)>0$
	\begin{align*}
		P(X=1,Y=1)                & > P(X=1)P(Y=1) \\
		\frac{P(X=1,Y=1)}{P(Y=1)} & > P(X=1)       \\
		P(X=1\mid Y=1)            & > P(X=1)
	\end{align*}
\end{proof}


\subsubsection{Coefficiente di correlazione}
\begin{defin}
	Il coefficiente di correlazione $\rho$ è un indice di correlazione tra variabili aleatorie discrete ed è così definito:
	\begin{equation*}
		\rho= \frac{\cov(X,Y)}{\sigma x\sigma y}
	\end{equation*}
\end{defin}

\section{Variabili aleatorie continue}
\begin{defin}[variabile aleatoria continua]
	Una variabile aleatoria $X$ è continua se e solo se non è discreta, ossia il suo supporto non è numerabile.
\end{defin}

\begin{defin}[funzione di densità di probabilità]
	Data una variabile aleatoria $X$, la funzione di densità di probabilità di $X$ è una funzione $f_X:\R\to\R^+$ integrabile e tale che:
	\begin{equation*}
		\int_{-\infty}^{+\infty} f_X(x)dx = 1
	\end{equation*}
\end{defin}

La probabilità di variabili aleatorie continue viene calcolata integrando la funzione di densità di probabilità:
\begin{equation*}
	\forall B\subseteq\R, P(X\in B)=\int_B f_X(x)dx
\end{equation*}
O, in un intervallo $[a,b]$:
\begin{equation*}
	P(a\leq X\leq b)=\int_a^b f_X(x)dx
\end{equation*}
che è equivalente a quella dell'intervallo $(a,b)$.

La proprietà della definizione non è altro che la conseguenza del primo assioma di Kolmogorov esteso al continuo:
\begin{equation*}
	P(X\in\R) = \int_{-\infty}^{+\infty} f_X(x)dx = \int_\R f_X(x)dx = 1
\end{equation*}

Per variabili aleatorie continue non ha senso calcolare la probabilità che una variabile assuma uno specifico valore, infatti:
\begin{equation*}
	\forall a\in\R \qquad P(X=a)=\int_a^a f_X(x)dx=0
\end{equation*}

In un certo intervallo che approssima $a$, ossia $[a-\frac{\varepsilon}{2},a+\frac{\varepsilon}{2}]$, invece:
\begin{equation*}
	P\left(a - \frac{\varepsilon}{2}\leq X\leq a + \frac{\varepsilon}{2}\right) = \int_{a - \frac{\varepsilon}{2}}^{a + \frac{\varepsilon}{2}} f_X(x)dx \approx \varepsilon f_X(a)
\end{equation*}

Calcolando l'integrale della funzione di densità in un intervallo del tipo $(-\infty,a)$ si calcola di fatto il valore della funzione di ripartizione in $a$:
\begin{equation*}
	\int_{-\infty}^a f_X(x)dx = P(X\leq a) = F_X(a) \\
\end{equation*}
Per il teorema fondamentale del calcolo integrale, da ciò deriva che la funzione di densità di probabilità è la derivata della funzione di ripartizione:
\begin{equation*}
	\frac{d}{dx}F_X(x)=f(x)
\end{equation*}

\subsection{Valore atteso}
\begin{defin}
	Il valore atteso per variabili aleatorie continue è così definito:
	\begin{equation*}
		\ev{X}=\int_{-\infty}^{+\infty} x f_X(x)dx
	\end{equation*}
\end{defin}

\begin{prop} \label{prop:valatnonneg}
	Per variabili aleatorie che non assumono specificazioni negative:
	\begin{equation*}
		\forall x<0 \qquad f_X(x)=0
	\end{equation*}
	vale
	\begin{equation*}
		\int_0^{+\infty} 1-F_X(x)dx = \ev{X}
	\end{equation*}
\end{prop}

\subsection{Disuguaglianza di Markov}
\begin{teor}[Disuguaglianza di Markov] \label{teor:markov}
	Sia $X$ una variabile aleatoria non negativa. Allora per ogni $a$ reale positivo:
	\begin{equation*}
		P(X\geq a) \leq \frac{\ev{X}}{a}
	\end{equation*}
\end{teor}
\begin{proof}
	\begin{align*}
		\ev{X} & = \int_0^{+\infty} x f_X(x) dx                  \bc{variabile non negativa} \\
		       & = \underbrace{\int_0^a x f_X(x) dx}_{\geq 0} + \int_a^{+\infty} x f_X(x) dx \\
		       & \geq \int_a^{+\infty} x f_X(x) dx                                           \\
		       & \geq \int_a^{+\infty} a f_X(x) dx                  \bc{in quanto $x\geq a$} \\
		       & = a \int_a^{+\infty} f_X(x) dx                                              \\
		       & = a P(X\geq a)
	\end{align*}
	ergo
	\begin{equation*}
		\frac{\ev{X}}{a}\geq P(X\geq a)
	\end{equation*}
\end{proof}



\subsection{Disuguaglianza di Tchebyshev}
La disuguaglianza di Tchebyshev dà una limitazione superiore alla probabilità che l'esito di una variabile aleatoria si discosti dal suo valore atteso di una quantità maggiore o uguale a una soglia scelta.
\begin{teor}[Disuguaglianza di Tchebyshev]
	Sia $X$ una variabile aleatoria, con $\ev{X}=\mu, \var(X)=\sigma^2$. Allora:
	\begin{equation*}
		\forall r>0\qquad P(\abs{X-\mu}\geq r)\leq \frac{\sigma^2}{r^2}
	\end{equation*}
\end{teor}
\begin{proof}
	\begin{equation*}
		\abs{X-\mu}\geq r \Leftrightarrow (X-\mu)^2 \geq r^2
	\end{equation*}
	Passando alle probabilità:
	\begin{align*}
		P(\abs{X-\mu}\geq r) & = P((X-\mu)^2 \geq r^2)                                       \\
		                     & \leq \frac{\ev{(X-\mu)^2}}{r^2} \bc{disuguaglianza di Markov} \\
		                     & =\frac{\var(X)}{r^2}
	\end{align*}
	Ergo:
	\begin{equation*}
		P(\abs{X-\mu}\geq r)\leq \frac{\sigma^2}{r^2}
	\end{equation*}
\end{proof}

Un'interessante applicazione della disuguaglianza di Tchebyshev è quella che riguarda la deviazione standard, che esprime l'andamento della probabilità allontanandosi dal valore atteso di quantità ripetute della deviazione standard:
\begin{equation*}
	P(\abs{X-\mu}\geq k\sigma)\leq\frac{1}{k^2}
\end{equation*}

%% Copyright (C) 2021 Alessandro Clerici Lorenzini
%
% This work may be distributed and/or modified under the
% conditions of the LaTeX Project Public License, either version 1.3
% of this license or (at your option) any later version.
% The latest version of this license is in
%   http://www.latex-project.org/lppl.txt
% and version 1.3 or later is part of all distributions of LaTeX
% version 2005/12/01 or later.
%
% This work has the LPPL maintenance status `maintained'.
%
% The Current Maintainer of this work is Alessandro Clerici Lorenzini
%
% This work consists of the files listed in work.txt


% TODO: valutare le funzioni (ripartizione e massa) nel dominio negativo
\section{Modelli}
I modelli di distribuzione consistono in risultati notevoli corrispondenti a variabili aleatorie che rispettano determinate definizioni ricorrenti.

% TODO: bisognerebbe aggiungere anche il supporto, tuttavia non riesco a sistemare la geometria della pagina in un modo soddisfacente che non causi warning.
% TODO: per quelli non visti, è sufficiente fare le somme delle funzioni di massa. Per il problema di spazio già citato, per ora sono omesse
% TODO: controllare la mancanza di funzioni indicatrici
\begin{sidewaystable}
	\centering
	\begin{tabular}{llllll}
		\toprule
		\bfseries Modello           & \bfseries Parametri   & \bfseries F. di massa/densità                                    & \bfseries F. di ripartizione                                                                                          & \bfseries V. atteso & \bfseries Varianza    \\
		\midrule
		\bfseries Bernoulli         & $X\sim B(p)$          & $p^x(1-p)^{1-x}I_{\{0,1\}}(x)$                                   & $\begin{cases}0\quad & x<0\\1-p\quad & 0\leq x<1\\1\quad & x\leq 1\end{cases}$                                                                                           & $p$                 & $p(1-p)$              \\[5ex]
		\bfseries Binomiale         & $X\sim B(n,p)$        & $\displaystyle\binom{n}{i} p^i(1-p)^{n-i}$                       & $\begin{cases}\sum_{i=0}^{\floor{x}}\binom{n}{i} p^i (1-p)^{n-i}\quad & x\leq n \\1 & x>n \end{cases}$                                                                                           & $np$                & $np(1-p)$             \\[3ex]
		\bfseries Uniforme discreto & $X\sim U(n)$          & $\dfrac{1}{n} I_{\{1,\dots,n\}}(i)$                              & $\dfrac{\floor{x}}{n}\cdot I_{\{1,\dots,n\}}+I_{\{n,\dots,+\infty\}}$                                                 & $\dfrac{n+1}{2}$    & $\dfrac{n^2-1}{12}$   \\[2ex]
		\bfseries Geometrico        & $X\sim G(p)$          & $(1-p)^ip ~ I_\N (i)$                                            & $1 - (1-p)^{\floor{x}+1}  $                                                                                           & $ \dfrac{1-p}{p}$   & $\dfrac{1-p}{p^2}$    \\[1ex]
		\bfseries Poisson           & $X\sim P(\lambda)$    & $e^{-\lambda}\frac{\lambda^i}{i!} ~ I_\N(i)$                     & [non visto]                                                                                                           & $\lambda$           & $\lambda$             \\[2ex]
		\bfseries Ipergeometrico    & $X\sim ?(N,M,n)$      & $\dfrac{\binom{N}{i}\binom{M}{n-i}}{\binom{N+M}{n}}$             & [non visto]                                                                                                           & $np$                & $\dfrac{NM}{(N+M)^2}$ \\[2ex]
		\bfseries Uniforme continuo & $X\sim U([a,b])$      & $\dfrac{1}{b-a} I[a,b](x)$                                       & $ \dfrac{x-a}{b-a}\cdot I_{[a,b]}(x)+I_{(b,+\infty)}(x)$                                                              & $\dfrac{b+a}{2}$    & $\dfrac{(b-a)^2}{12}$ \\[3ex]
		\bfseries Esponenziale      & $X\sim E(\lambda)$    & $\lambda e^{-\lambda x} I_{\R^+}(x)$                             & $1-e^{-\lambda x}$                                                                                                    & $\frac{1}{\lambda}$ & $\frac{1}{\lambda^2}$ \\[1ex]
		\bfseries Gaussiano         & $X\sim N(\mu,\sigma)$ & $\dfrac{1}{\sqrt{2\pi}~\sigma}e^{-\dfrac{(x-\mu)^2}{2\sigma^2}}$ & $\displaystyle\int_{-\infty}^x \dfrac{1}{\sqrt{2\pi}~\sigma} e^{-\frac{1}{2}\left(\dfrac{y-\mu}{\sigma}\right)^2} dy$ & $\mu$               & $\sigma^2$            \\
		\bottomrule
	\end{tabular}
	\caption{Tabella riassuntiva dei modelli di distribuzione.}
\end{sidewaystable}

\subsection{Modello bernoulliano}
Il modello bernoulliano impone alle sue variabili aleatorie di avere specificazione binaria, cioè $D_X=\{0,1\}$ per qualunque variabile $X$.

Una variabile aleatoria bernoulliana di probabilità di successo ($X=1$) $p$ si indica con
\begin{equation*}
	X\sim B(p)
\end{equation*}


\subsubsection{Funzione di massa di probabilità}
La funzione di massa di probabilità di una variabile aleatoria bernoulliana è
\begin{equation*}
	p_X(x)=P(X=x)=p^x(1-p)^{1-x}I_{\{0,1\}}(x)=\begin{cases}
		1-p \qquad & x=0               \\
		p \qquad   & x=1               \\
		0 \qquad   & \text{altrimenti}
	\end{cases}
\end{equation*}


\subsubsection{Valore atteso}
In analogia con la proprietà \ref{prop:indvalatt}, il valore atteso di una variabile aleatoria bernoulliana è $p$:
\begin{equation*}
	\ev{X} = p
\end{equation*}
E ovviamente
\begin{equation*}
	\ev{X^2}=\ev{X}
\end{equation*}


\subsubsection{Varianza}
In analogia con la proprietà \ref{prop:indvar}, la varianza di una variabile aleatoria bernoulliana è $p(1-p)$:
\begin{equation*}
	\var(X) = p(1-p)
\end{equation*}


\subsubsection{Funzione di ripartizione}
La funzione di ripartizione per variabili bernoulliane è ovviamente uguale a $0$ per $x<0$, $1-p$ per $0\leq x<1$ e $1$ per $x\geq 1$.



\subsection{Modello binomiale}
Il modello binomiale consiste in $n$ ripetizioni di un esperimento bernoulliano di probabilità $p$. Una variabile aleatoria binomiale corrisponde al numero di successi tra gli $n$ esperimenti.
\begin{equation*}
	X \sim B(n, p)\qquad D_X=\{0,1,\dots,n\}
\end{equation*}


\subsubsection{Funzione di massa di probabilità}
Per definizione:
\begin{equation*}
	p_X(i) = P(X=i)
\end{equation*}
Tale probabilità è l'intersezione degli eventi indipendenti che consistono nel successo dal primo all'$i$-esimo e insuccesso dall'$i+1$-esimo all'$n$-esimo. In quanto probabilità di eventi indipendenti, essa può essere espressa come il prodotto delle singole probabilità. Ognuna delle probabilità di successo, per come è costruito il modello binomiale, è $p$, mentre ognuna delle probabilità di insuccesso è $1-p$. Ogni combinazione in cui $i$ esperimenti hanno successo e $n-i$ falliscono è valida, perciò:
\begin{equation}
	P(X=i) = \binom{n}{i} p^i(1-p)^{n-i}
\end{equation}

Concorde con la \eqref{eq:sommamassa}, la somma delle immagini della funzione di massa di probabilità è 1:
\begin{equation*}
	\sum_{i=1}^n p_X(i) = \sum_{i=1}^n \binom{n}{i} p^i(1-p)^{n-i} = (p+1-p)^n = 1
\end{equation*}


\subsubsection{Valore atteso}
Essendo ogni variabile aleatoria binomiale la somma di variabili aleatorie bernoulliane:
\begin{equation}
	\ev{X} = \ev{\sum_{i=0}^n X_i} = \sum_{i=0}^n \ev{X_i} = \sum_{i=0}^n p = np
\end{equation}


\subsubsection{Varianza}
Essendo le componenti bernoulliane indipendenti:
\begin{equation}
	\var(X)=\sum_{i=1}^n \var(X_i) = \sum_{i=1}^n p(1-p) = np(1-p)
\end{equation}


\subsubsection{Funzione di ripartizione}
Per calcolare con un'unica formula la funzione di ripartizione si aggiungono due funzioni indicatrici, che agiscono se $x>n$:
\begin{align}
	F_X(x) & = P(X\leq x)                                                                           \nonumber            \\
	       & = I_{(n,+\infty)}(x) + I_{[0,n]}(x) \sum_{i=0}^{\floor{x}}\binom{n}{i} p^i (1-p)^{n-i} \label{eq:ripabinom} \\
	       & = \begin{cases}
		\sum_{i=0}^{\floor{x}}\binom{n}{i} p^i (1-p)^{n-i}\quad & x\leq n \\
		1                                                       & x>n     \\
	\end{cases} \nonumber
\end{align}

\subsubsection{Relazioni tra variabili binomiali}
Siano $X_1$ e $X_2$ due variabili aleatorie definite su modelli binomiali che differiscono solo per il numero di esperimenti:
\begin{align*}
	 & X_1\sim B(n, p)\quad & X_1=\sum_{i=1}^n X_{1,i}\qquad & X_{1,i}\sim B(p)~\forall i\in\{1,\dots,n\} \\
	 & X_2\sim B(m, p)\quad & X_1=\sum_{j=1}^m X_{2,j}\qquad & X_{2,j}\sim B(p)~\forall j\in\{1,\dots,m\}
\end{align*}

\noindent
Se $X_1$ e $X_2$ sono indipendenti, allora:
\begin{equation*}
	X_1+X_2 = \sum_{i=1}^n X_{1,i} + \sum_{j=1}^m X_{2,j} = \sum_{i=1}^{n+m} Y_i = Y
\end{equation*}
dove $Y\sim B(n+m, p)$.



\subsection{Modello uniforme discreto} \label{subsec:unifdisc}
Nel modello uniforme discreto le variabili aleatorie consistono nell'esito di un esperimento da $n$ esiti possibili equiprobabili:
\begin{equation*}
	X\sim U(n)
\end{equation*}


\subsubsection{Funzione di massa di probabilità}
Essendo gli $n$ esiti equiprobabili, la funzione di massa di probabilità assume il valore $\frac{1}{n}$ per tutti i valori dell'input compresi tra gli esiti (numerati qui da $1$ a $n$):
\begin{equation*}
	p_X(i) = P(X=i) = \frac{1}{n} I_{\{1,\dots,n\}}(i)
\end{equation*}


\subsubsection{Funzione di ripartizione}
\begin{equation*}
	\forall x\leq n\qquad F_X(x) = P(X\leq x) = \sum_{i=1}^{\floor{x}} P(X=i) = \sum{i=1}^{\floor{x}} \frac{1}{n} = \frac{\floor{x}}{n}
\end{equation*}
Come per la \eqref{eq:ripabinom}, si aggiungono funzioni indicatrici per regolare il valore oltre $n$:
\begin{equation}
	F_X(x)=\frac{\floor{x}}{n} I_{[1,n]}(x)+I_{(n,+\infty)}(x)
\end{equation}


\subsubsection{Valore atteso}
Banalmente, usando la definizione:
\begin{equation} \label{eq:disunvalat}
	\ev{X} = \sum_{i=1}^n i P(X=i) = \frac{1}{n} \sum_{i=1}^n i = \frac{1}{n}\frac{n(n+1)}{2} = \frac{n+1}{2}
\end{equation}


\subsubsection{Varianza}
Usando la definizione equivalente di varianza e quanto appena calcolato:
\begin{align*}
	\var(X) & = \ev{X^2} - \ev{X}^2 =                                       \bc{proprietà \ref{prop:varalt}}           \\
	        & = \sum_{i=1}^n i^2 P(X=i) - \left(\frac{n+1}{2}\right)^2      \bc{valore atteso e \eqref{eq:disunvalat}} \\
	        & = \frac{1}{n} \sum_{i=1}^n i^2 - \left(\frac{n+1}{2}\right)^2 \bc{ipotesi di equiprobabilità}            \\
	        & = \frac{(n+1)(2n+1)}{6}-\left(\frac{n+1}{2}\right)^2          \bc{somma notevole}                        \\
	        & = (n+1)\left(\frac{2n+1}{6}+\frac{n+1}{4}\right)              \bc{raccogliendo $n+1$}                    \\
	        & = (n+1)\left(\frac{n-1}{12}\right)                                                                       \\
	        & = \frac{n^2-1}{12}
\end{align*}



\subsection{Modello geometrico}
Una variabile aleatoria geometrica corrisponde al numero di insuccessi prima del primo successo in una sequenza di esperimenti bernoulliani con lo stesso parametro $p$ e tra loro indipendenti.

Per $p=0$ non si ottiene mai un successo, pertanto la variabile geometrica non è definita. Per $p=1$ la variabile assume necessariamente il valore $0$.

Il supporto di una variabile geometrica è l'insieme dei naturali.

\begin{equation*}
	X\sim G(p)\qquad D_X=\N
\end{equation*}


\subsubsection{Funzione di massa di probabilità}
La funzione di massa di probabilità in $i$ è uguale alla probabilità di insuccesso per ognuna delle $i$ ripetizioni indipendenti per la probabilità di successo della $i+1$-esima.

Come sempre si aggiunge una funzione indicatrice per aggiustare il dominio di $F_X$:
\begin{equation} \label{eq:geommasprob}
	F_X(i) = P(X=i) = (1-p)^ip ~ I_\N (i)
\end{equation}

La somma delle immagini della funzione di massa di probabilità converge, come dovrebbe, a $1$:
\begin{align*}
	\sum_{i=0}^{+\infty} P(X=i) & = \sum_{i=0}^{+\infty} p(1-p)^i \\
	                            & = p\sum_{i=0}^{+\infty} (1-p)^i \\
	                            & = p\frac{1}{1-(1-p)}            \\
	                            & = 1
\end{align*}


\subsubsection{Valore atteso}
Tramite la definizione:
\begin{align*}
	\ev{X} & = \sum_{i=0}^{+\infty} i P(X=i)                    \bc{definizione \ref{def:valatt}}         \\
	       & = \sum_{i=0}^{+\infty} ip(1-p)^i                   \bc{\eqref{eq:geommasprob}}               \\
	       & = p(1-p)\sum_{i=0}^{+\infty} i(1-p)^{i-1}                                                    \\
	       & = -p(1-p) \frac{d}{dx} \sum_{i=0}^{+\infty} (1-p)^i \bc{derivata di $(1-p)^i$ e della somma} \\
	       & = -p(1-p) \frac{d}{dx} \frac{1}{p}                  \bc{serie geometrica di ragione $1-p$}   \\
	       & = \frac{p(1-p)}{p^2} = \frac{1-p}{p}                 \bc{derivando}
\end{align*}


\subsubsection{Varianza}
Volendo usare la forma equivalente \eqref{eq:varalt} di cui alla proprietà \ref{prop:varalt}, si calcola innanzitutto il valore atteso del quadrato della variabile:
\begin{align*}
	\ev{X^2} & = \sum_{i=0}^{+\infty} i^2 p(1-p)^i                                                                                                          \\
	         & = p(1-p) \sum_{i=0}^{+\infty} i^2 (1-p)^{i-1}                                                                                                \\
	         & = -p(1-p) \sum_{i=0}^{+\infty} i \frac{d}{dp}(1-p)^i \bc{derivata di $(1-p)^i$}                                                              \\
	         & = -p(1-p) \frac{d}{dp}\sum_{i=0}^{+\infty} i (1-p)^i \bc{\parbox{42.5mm}{prodotto di una derivata per una costante e derivata di una somma}} \\
	         & = -p(1-p) \frac{d}{dp}(1-p)\sum_{i=0}^{+\infty} i (1-p)^{i-1}                                                                                \\
	         & = p(1-p) \frac{d}{dp}(1-p) \frac{d}{dp} \sum_{i=0}^{+\infty} (1-p)^i                                                                         \\
	         & = -p(1-p) \frac{d}{dp} \frac{1-p}{p^2}                                                                                                       \\
	         & = -p(1-p) \frac{-p^2-2p(1-p)}{p^4}                                                                                                           \\
	         & = (1-p) \frac{p+2(1-p)}{p^2}                                                                                                                 \\
	         & = \frac{(1-p)(2-p)}{p^2}
\end{align*}
Da cui:
\begin{align*}
	\var(X) & = \ev{X^2}-\ev{X}^2                                     \\
	        & = \frac{(1-p)(2-p)}{p^2} - \left(\frac{1-p}{p}\right)^2 \\
	        & = \frac{(1-p)((2-p)-(1-p))}{p^2}                        \\
	        & = \frac{1-p}{p^2}
\end{align*}


\subsubsection{Funzione di ripartizione}
\begin{align*}
	F_X(n) & = P(X\leq n) = 1-P(X>n) =                                                                                                 \\
	       & = 1 - \sum_{i=n+1}^{+\infty} P(X=i)                                                                                       \\
	       & = 1 - \sum_{i=n+1}^{+\infty} p(1-p)^i                                                                                     \\
	       & = 1 - p(1-p)^{n+1} \sum_{i=n+1}^{+\infty} (1-p)^{i-(n+1)} \justif{~}{moltiplicando per $\frac{(1-p)^{n+1}}{(1-p)^{n+1}}$} \\
	       & = 1 - p(1-p)^{n+1} \sum_{i=0}^{+\infty} (1-p)^i \justif{~}{sostituzione: $i=i-(n+1)$}                                     \\
	       & = 1 - p(1-p)^{n+1} \frac{1}{1-(1-p)}                                                                                      \\
	       & = 1 - (1-p)^{n+1}                                                                                                         \\
\end{align*}
Questo risultato è in realtà banale se si applica il concetto semantico alla variabile geometrica.

\begin{equation*}
	F_X(x) = P(X\leq x) = 1-P(X>x) = 1-(1-p)^{\floor{x}+1}
\end{equation*}

\subsubsection{Assenza di memoria} \label{geom-assmem}
Come si può intuire, la probabilità di costante insuccesso all'$i+j$-esimo esperimento non è condizionata dalla probabilità di costante insuccesso all'$i$-esimo. Questo risultato prende il nome di assenza di memoria.
\begin{align*}
	P(X\geq i+j \mid X\geq i) & = \frac{P(X\geq i+j \cap X\geq i)}{P(X\geq i)} \\
	                          & = \frac{P(X\geq i+j)}{P(X\geq i)}              \\
	                          & = \frac{(1-p)^{i+j}}{(1-p)^i}                  \\
	                          & = (1-p)^j                                      \\
	                          & = P(X\geq j)
\end{align*}



\subsection{Modello di Poisson}
\begin{equation*}
	X\sim P(\lambda) \qquad D_X = \N \qquad \lambda>0
\end{equation*}

\subsubsection{Funzione di massa di probabilità}
\begin{equation*}
	P_X(i) = P(X=i) = e^{-\lambda}\frac{\lambda^i}{i!} ~ I_\N(i)
\end{equation*}

Ancora una volta la somma delle immagini della funzione di massa di probabilità converge a $1$:
\begin{equation} \label{eq:poisummas}
	\sum_{i=0}^{+\infty} e^{-\lambda}\frac{\lambda^i}{i!} = e^{-\lambda}\sum_{i=0}^{+\infty} \frac{\lambda^i}{i!} = e^{-\lambda}e^\lambda = 1
\end{equation}


\subsubsection{Valore atteso}
Tramite la definizione:
\begin{align}
	\ev{X} & = \sum_{i=0}^{+\infty} i P(X=i) = \sum_{i=1}^{+\infty} i P(X=i) \bc{definizione \ref{def:valatt}} \nonumber \\
	       & = \sum_{i=1}^{+\infty} i e^{-\lambda}\frac{\lambda^i}{i!} \label{eq:poinotevolelambda}                      \\
	       & = \lambda e^{-\lambda} \sum_{i=1}^{+\infty} \frac{\lambda^{i-1}}{(i-1)!} \nonumber                          \\
	       & = \lambda e^{-\lambda} \sum_{i=0}^{+\infty}\frac{\lambda^i}{i!} \bc{sostituzione: $i=i-1$} \nonumber        \\
	       & = \lambda e^{-\lambda} e^\lambda \nonumber                                                                  \\
	       & = \lambda
\end{align}


\subsubsection{Varianza}
Volendo usare la forma equivalente \eqref{eq:varalt}, si calcola innanzitutto il valore atteso di $X^2$:
\begin{align*}
	\ev{X^2} & = \sum_{i=1}^{+\infty} i^2 e^{-\lambda}\frac{\lambda^i}{i!}                                                                                                                              \\
	         & = \sum_{i=1}^{+\infty} i e^{-\lambda} \frac{\lambda^i}{(i-1)!}                                                                                                                           \\
	         & = \lambda\sum_{i=1}^{+\infty} i e^{-\lambda}\frac{\lambda^{i-1}}{(i-1)!}                                                                                                                 \\
	         & = \lambda\sum_{i=1}^{+\infty} (i-1+1) e^{-\lambda}\frac{\lambda^{i-1}}{(i-1)!}                                                                                                           \\
	         & = \lambda\sum_{i=1}^{+\infty}\left((i-1)e^{-\lambda}\frac{\lambda^{i-1}}{(i-1)!}+e^{-\lambda}\frac{\lambda^{i-1}}{(i-1)!}\right)                                                         \\
	         & = \lambda\sum_{i=1}^{+\infty} (i-1)e^{-\lambda}\frac{\lambda^{i-1}}{(i-1)!} + \lambda\sum_{i=1}^{+\infty} + e^{-\lambda}\frac{\lambda^{i-1}}{(i-1)!}                                     \\
	         & = \lambda\underbrace{\sum_{i=0}^{+\infty} ie^{-\lambda}\frac{\lambda^i}{i!}}_{\lambda} + \lambda \underbrace{\sum_{i=0}^{+\infty} e^{-\lambda}\frac{\lambda^i}{i!}}_{1} \bc{con $i=i-1$} \\
	         & = \lambda^2 + \lambda \bc{\eqref{eq:poinotevolelambda} e \eqref{eq:poisummas}}
\end{align*}
Ergo
\begin{align}
	\var(X) & = \ev{X^2} - \ev{X}^2         \nonumber \\
	        & = \lambda^2+\lambda-\lambda^2 \nonumber \\
	        & = \lambda
\end{align}

\subsubsection{Approssimazione del modello binomiale} \label{subsub:binompois}
Il modello di Poisson è strettamente legato al modello binomiale. Infatti, se il prodotto dei parametri di una binomiale è costante, per $n$ grandi essa è ben approssimata da una variabile di Poisson che ha come parametro tale prodotto:
\begin{equation*}
	X\sim B(n, p)\qquad\text{con }np=\lambda
\end{equation*}
Per $n\to+\infty$:
\begin{align*}
	P(X=i) & = \binom{n}{i} p^i (1-p)^{n-i}                                                                                                                                                                                                                                                \\
	       & = \binom{n}{i}\left(\frac{\lambda}{n}\right)^i\left(1-\frac{\lambda}{n}\right)^{n-i}                                                                                                                                                                                          \\
	       & = \frac{n(n-1) \dots (n-i+1)}{i!} \cdot \frac{\lambda^i}{n^i} \left( 1 - \frac{\lambda}{n} \right)^{n-i}                                                                                                                                                                      \\
	       & = \frac{n(n-1) \dots (n-i+1)}{n^i} \cdot \frac{\lambda^i}{i!} \left( 1 - \frac{\lambda}{n} \right)^{n-i}                                                                                                                                                                      \\
	       & = \underbrace{\frac{n}{n}}_{\to 1} \cdot \underbrace{\frac{n-1}{n}}_{\to 1} \dots \underbrace{\frac{n-i+1}{n}}_{\to 1} \cdot \frac{\lambda^i}{i!} \cdot \underbrace{\frac{\left( 1 - \frac{\lambda}{n} \right)^n}{\left( 1 - \frac{\lambda}{n} \right)^i}}_{\to e^{-\lambda}} \\
	       & \to \frac{\lambda^i}{i!} e^{-\lambda}
\end{align*}



\subsection{Modello ipergeometrico}
Il modello ipergeometrico descrive il classico problema dell'urna. Dati $N$ oggetti funzionanti e $M$ oggetti difettosi, sia $n$ il numero di estrazioni senza reimmissione. La variabile aleatoria ipergeometrica X è il numero di oggetti funzionanti nelle $n$ estrazioni.
\begin{equation*}
	X\sim ?(?)
\end{equation*}

Il modello è valido solo se $P(X=0)=0$.


\subsubsection{Funzione di massa di probabilità}
Il numero di casi possibili sono le combinazioni di $n$ estrazioni senza reimmissione da un gruppo di $N+M$. Il numero di casi favorevoli si può calcolare applicando il principio fondamentale del calcolo combinatorio.
\begin{equation*}
	P(X=i) = \frac{\binom{N}{i}\binom{M}{n-i}}{\binom{N+M}{n}}
\end{equation*}


\subsubsection{Valore atteso}
Al fine di calcolare il valore atteso si sfrutta un approccio decomposizionale: si introducono $n$ variabili aleatorie $X_i$, ognuna legata a un'estrazione, tali che
\begin{equation*}
	X_i = \begin{cases}
		1 & \text{l'$i$-esimo oggetto estratto funziona} \\
		0 & \text{altrimenti}
	\end{cases}
\end{equation*}

Per tali variabili vale
\begin{equation*}
	P(X_i=1) = \frac{N}{N+M} =: p = \ev{X_i}
\end{equation*}

Da cui
\begin{align*}
	\ev{X} & = \ev{\sum_{i=1}^n X_i} \\
	       & = \sum_{i=1}^n \ev{X_i} \\
	       & = np
\end{align*}


\subsubsection{Varianza}
Applicando la proprietà \ref{prop:varalt} si calcola la varianza delle singole $X_i$:
\begin{align*}
	\var(X_i) & = \ev{X_i^2}-\ev{X_i}^2                                   \\
	          & = \ev{X_i}(1-\ev{X_i})          \bc{idempotenza di $X_i$} \\
	          & = \frac{N}{N+M} + \frac{M}{N+M}                           \\
	          & = \frac{NM}{(N+M)^2}
\end{align*}

Essendo le variabili $X_i$ non indipendenti, la varianza della loro somma non è uguale alla somma delle varianze. Si può comunque applicare la proprietà \ref{prop:varsumcov} e passare per le covarianze:
\begin{align*}
	\cov(X_i,X_j) & = \ev{X_i X_j} - \ev{X_i}\ev{X_j}                                \\
	              & = \ev{X_i=1\cap X_j=1} - \left(\frac{N}{N+M}\right)^2            \\
	              & = P(X_j=1 \mid X_i=1)P(X_i=1) - \left(\frac{N}{N+M}\right)^2     \\
	              & = \frac{N-1}{N+M-1} \frac{N}{N+M} - \left(\frac{N}{N+M}\right)^2 \\
	              & = \frac{N}{N+M} \left(\frac{N-1}{N+M-1} - \frac{N}{N+M}\right)   \\
	              & = \frac{-NM}{(N+M-1)(N+M)^2}
\end{align*}
Ergo
\begin{align*}
	\var(X) & = \sum_{i=1}^n \var(X_i) + \sum_{i\neq j}^n \cov(X_i,X_j)     \\
	        & = n\frac{NM}{(N+M)^2} - n(n-1)\cdot\frac{-NM}{(N+M-1)(N+M)^2} \\
	        & = n\frac{NM}{(N+M)^2}\left(1-(n-1)\frac{1}{N+M-1}\right)      \\
	        & = np(1-p)\left(1-\frac{n-1}{N+M-1}\right)
\end{align*}
Per $N+M\to+\infty$ il modello si semplifica in un modello binomiale:
\begin{equation*}
	\to np(1-p)
\end{equation*}


\subsection{Modello uniforme continuo}
Il modello uniforme continuo estende al continuo i concetti visti alla sezione \ref{subsec:unifdisc} per il modello uniforme discreto e viene determinato da un intervallo equivalentemente aperto o chiuso:
\begin{equation*}
	X \sim U([a,b])
\end{equation*}


\subsubsection{Funzione di densità di probabilità}
\begin{equation*}
	f(X)(x) = \frac{1}{b-a} I[a,b](x)
\end{equation*}
essendo la densità costante, per $I\subseteq[a,b]$:
\begin{equation*}
	P(X\in I) = \frac{|I|}{b-a}
\end{equation*}
dove $|I|$ è la somma delle ampiezze $p-q$ degli intervalli disgiunti $[p,q]$ da cui $I$ è composto.

Come da definizione, integrando nell'intero $\R$ la funzione di densità di probabilità si ottiene $1$:
\begin{equation*}
	\int_{-\infty}^{+\infty}f_X(x) = \int_a^b\frac{1}{b-a}dx = \frac{1}{b-a} \eval{x}{a}{b} = 1
\end{equation*}


\subsubsection{Funzione di ripartizione}
Per definizione la funzione di ripartizione è la funzione integrale della funzione di densità:
\begin{align*}
	F_X(x) & = P(X\leq x)                   \\
	       & = \int_{-\infty}^x f_X(u)du    \\
	       & = \int_a^x \frac{1}{b-a}du     \\
	       & = \frac{1}{b-a} \eval{u}{a}{x} \\
	       & = \frac{x-a}{b-a}
\end{align*}

Come sempre funzioni indicatrici aggiustano il risultato per punti non appartenenti all'intervallo:
\begin{equation*}
	F_X(x) = \frac{x-a}{b-a} I_{[a,b]}(x) + I_{(b,+\infty)}(x)
\end{equation*}


\subsubsection{Valore atteso}
Applicando la definizione:
\begin{align*}
	\ev{X} & = \int_a^b x f_X(x)dx                      \\
	       & = \frac{1}{b-a} \int_a^b x ~ dx            \\
	       & = \frac{1}{b-a} \eval{\frac{x^2}{2}}{a}{x} \\
	       & = \frac{1}{b-a} \cdot \frac{b^2-a^2}{2}    \\
	       & = \frac{b+a}{2}
\end{align*}


\subsubsection{Varianza}
Come di consueto si intende applicare la proprietà \ref{prop:varalt}, pertanto si calcola innanzitutto il valore atteso di $X^2$:
\begin{align*}
	\ev{X^2} & = \int_a^b x^2 f_X(x)dx                    \\
	         & = \frac{1}{b-a}\int_a^b x^2 ~ dx           \\
	         & = \frac{1}{b-a} \eval{\frac{x^3}{3}}{a}{b} \\
	         & = \frac{b^3-a^3}{3(b-a)}                   \\
	         & = \frac{a^2+ab+b^2}{3}
\end{align*}

E infine:
\begin{align*}
	\var(X) & = \ev{X^2} - \ev{X}^2                      \\
	        & = \frac{a^2+ab+b^2}{3} - \frac{(a+b)^2}{4} \\
	        & = \frac{(b-a)^2}{12}
\end{align*}


\subsection{Modello esponenziale}
Nel modello esponenziale la funzione di densità è esponenziale.
\begin{equation*}
	X\sim E(\lambda) \qquad \lambda\in\R^+ \quad D_X=\R^+
\end{equation*}


\subsubsection{Funzione di densità di probabilità}
\begin{equation*}
	f_X(x) = \lambda e^{-\lambda x} I_{\R^+}(x)
\end{equation*}

Il modello esponenziale si usa per modellare il tempo che intercorre tra due eventi.

La funzione di densità rispetta la definizione, infatti:
\begin{align*}
	\int_0^{+\infty} f_X(x)dx & = \int_0^{+\infty} \lambda e^{-\lambda x} dx       \\
	                          & = \int_0^{+\infty} e^{-y}dy \bc{con $y=\lambda x$} \\
	                          & = \eval{-e^{-y}}{0}{+\infty}                       \\
	                          & = 0 + e^{-0}                                       \\
	                          & = 1
\end{align*}


\subsubsection{Funzione di ripartizione}
Applicando la definizione di funzione di ripartizione continua:
\begin{align*}
	F_X(x) & = \int_0^x f_X(y)dy                                  \\
	       & = \int_0^x \lambda e^{\lambda y}dy                   \\
	       & = \int_0^{\lambda x} e^{-z}dz \bc{con $z=\lambda y$} \\
	       & = \eval{-e^{-z}}{0}{\lambda x}                       \\
	       & = -e^{-\lambda x} + e^0                              \\
	       & = 1-e^{-\lambda x}
\end{align*}
Aggiungendo una funzione indicatrice:
\begin{equation}
	F_X(x) = (1 - e^{-\lambda x})I_{\R^+}(x)
\end{equation}


\subsubsection{Valore atteso}
Applicando la definizione:
\begin{align*}
	\ev{X} & = \int_0^{+\infty} x f_X(x) dx                                                             \\
	       & = \int_0^{+\infty} x\lambda e^{-\lambda x}dx                                               \\
	       & = \eval{-x e^{-\lambda x}}{0}{+\infty} + \int_0^{+\infty} e^{-\lambda x} dx \bc{per parti} \\
	       & = \int_0^{+\infty} e^{-\lambda x} dx                                                       \\
	       & = \frac{1}{\lambda} \underbrace{\int_0^{+\infty} \lambda e^{-\lambda x}}_{1}               \\
	       & = \frac{1}{\lambda} \bc{proprietà di $f_X$}
\end{align*}


\subsubsection{Varianza}
Volendo usare la forma equivalente \eqref{eq:varalt}, si calcola innanzitutto il valore atteso di $X^2$:
\begin{align*}
	\ev{X^2} & = \int_0^{+\infty} x^2\lambda e^{-\lambda x} dx                                 \\
	         & = \eval{-x^2 e^{-\lambda x}}{0}{+\infty} + \int_0^{+\infty} 2xe^{-\lambda x} dx \\
	         & = 2\int_0^{+\infty} xe^{-\lambda x}dx                                           \\
	         & = \frac{2}{\lambda} \int_0^{+\infty} \lambda xe^{-\lambda x}dx                  \\
	         & = \frac{2}{\lambda} \ev{X} = \frac{2}{\lambda^2}
\end{align*}
Da cui:
\begin{align*}
	\var(X) & = \ev{X^2} - \ev{X}^2                                           \\
	        & = \frac{2}{\lambda^2}-\frac{1}{\lambda^2} = \frac{1}{\lambda^2}
\end{align*}


\subsubsection{Assenza di memoria}
Le variabili di modello esponenziale godono della proprietà di assenza di memoria (già vista per il modello geometrico al paragrafo \ref{geom-assmem}):
\begin{equation*}
	P(X>x) = 1 - F_X(x) = e^{-\lambda x}
\end{equation*}
Quindi:
\begin{align*}
	P(X>s+t) & = e^{-\lambda (s+t)}           \\
	         & = e^{-\lambda s}e^{-\lambda t} \\
	         & = P(X>s)P(X>t)
\end{align*}
Da cui
\begin{align*}
	P(X>s) & = \frac{P(X>s+t)}{P(X>t)}         \\
	       & = \frac{P(X>s+t\cap X>t)}{P(X>t)} \\
	       & = P(X>s+t\mid X>t)
\end{align*}


\subsection{Risultati notevoli sui modelli}
\begin{prop}
	Siano $X_1,\dots,X_n$ variabili aleatorie indipendenti e sia $Y$ il massimo degli $X_i$, ossia $Y:=\max_i X_i$. Allora:
	\begin{equation*}
		F_Y(x) = \prod_{i=1}^n F_{X_i}(x)
	\end{equation*}
	E per variabili indipendenti e identicamente distribuite (i.i.d.) secondo una funzione di ripartizione $F$:
	\begin{equation*}
		F_Y(x) = \prod_{i=1}^n F(x) = F(x)^n
	\end{equation*}
\end{prop}
\begin{proof}
	\begin{equation*}
		F_Y(x) = P(Y\leq x) = P(\max_i X_i\leq x) = P(\forall i X_i\leq x)
	\end{equation*}
	Dal momento che gli $X_i$ sono indipendenti, l'ultima probabilità è uguale al prodotto delle singole:
	\begin{equation*}
		= \prod_{i=1}^n P(X_i \leq x) = \prod_{i=1}^n F_{X_i}(x)
	\end{equation*}
	Nell'ulteriore ipotesi di variabili indipendenti e identicamente distribuite (i.i.d.) secondo una funzione di ripartizione $F$:
	\begin{equation*}
		= \prod_{i=1}^n F(x) = F(x)^n
	\end{equation*}
\end{proof}

\begin{prop} \label{prop:modnotmin}
	Siano $X_1,\dots,X_n$ variabili aleatorie indipendenti e sia $Z$ il minimo degli $X_i$, ossia $Z:=\min_i X_i$. Allora:
	\begin{equation*}
		F_Z(x) = 1 - \prod_{i=1}^n (1-F_{X_i}(x))
	\end{equation*}
	E per variabili indipendenti e identicamente distribuite (i.i.d.) secondo una funzione di ripartizione $F$:
	\begin{equation*}
		F_Z(x) = 1 - (1-F(x))^n
	\end{equation*}
\end{prop}
\begin{proof}
	\begin{equation*}
		F_Z(x) = 1 - P(Z>x) = 1 - P(\min X_i > x) = 1 - P(\forall i X_i > x)
	\end{equation*}
	Dal momento che gli $X_i$ sono indipendenti, l'ultima probabilità è uguale al prodotto delle singole:
	\begin{equation*}
		= 1 - \prod_{i=1}^n P(X_i > x) = 1 - \prod_{i=1}^n (1-F_{X_i}(x))
	\end{equation*}
	Nell'ulteriore ipotesi di variabili indipendenti e identicamente distribuite (i.i.d.) secondo una funzione di ripartizione $F$:
	\begin{equation*}
		= 1 - \prod_{i=1}^n (1-F(x)) = 1 - (1-F(x))^n
	\end{equation*}
\end{proof}

\begin{prop}
	Siano $X_1,\dots,X_n$ variabili aleatorie indipendenti e sia $Z$ il minimo degli $X_i$, ossia $Z:=\min_i X_i$. Se $X_i\sim E(\lambda_i)$ per ogni $i$, allora:
	\begin{equation*}
		Z\sim E\left(\sum_{i=1}^n \lambda_i\right)
	\end{equation*}
\end{prop}
\begin{proof}
	Essendo le variabili esponenziali le loro funzioni di ripartizione sono del tipo:
	\begin{equation*}
		F_{X_i}(x) = 1-e^{-\lambda_i x}
	\end{equation*}
	Per la proprietà \ref{prop:modnotmin}:
	\begin{align*}
		F_Z(x) & = 1 - \prod_{i=1}^n (1-F_{X_i}(x))   \\
		       & = 1 - \prod_{i=1}^n e^{-\lambda_i x} \\
		       & = 1 - e^{\sum_{i=1}^n -\lambda_i x}  \\
		       & = 1 - e^{-x \sum_{i=1}^n \lambda_i}  \\
	\end{align*}
	Chiamato $\lambda = \sum_{i=1}^n \lambda_i$, allora $Z\sim E(\lambda)$:
	\begin{equation*}
		F_Z(x) = 1 - e^{-\lambda x}
	\end{equation*}
\end{proof}

\begin{prop}
	Se $X\sim E(\lambda)$ e $Y:=cX$ con $c\in\R^+$, allora $Y$ è una variabile aleatoria esponenziale di parametro $\frac{\lambda}{c}$.
	\begin{equation*}
		F_Y(x) = 1 - e^{-\frac{\lambda}{c} x}
	\end{equation*}
\end{prop}
\begin{proof}
	\begin{align*}
		F_Y(x) & =                                  \\
		       & = P(Y \leq x)                      \\
		       & = P(Xc \leq x)                     \\
		       & = P\left(X \leq \frac{x}{c}\right) \\
		       & = F_X\left(\frac{x}{c}\right)      \\
		       & = 1 - e^{-\frac{\lambda}{c} x}     \\
	\end{align*}
\end{proof}


\subsection{Modello gaussiano}
Una variabile $X$ di modello gaussiano (o normale), è una variabile aleatoria continua definita da due parametri:
\begin{equation*}
	X\sim N(\mu, \sigma)\qquad \mu\in\R,\sigma\in\R^+
\end{equation*}


\subsubsection{Funzione di densità di probabilità}
\begin{equation*}
	f_X(x) = \frac{1}{\sqrt{2\pi}~\sigma}e^{-\dfrac{(x-\mu)^2}{2\sigma^2}}
\end{equation*}

Studiando la funzione di densità si verifica algebricamente la famosa forma \qt{a campana}:
\begin{align*}
	 & \bullet \lim_{x\to\pm\infty} f_X(x) = 0                                                 \\
	 & \bullet f'_x(x) = \frac{1}{\sqrt{2\pi}~\sigma^3}e^{-\frac{(x-\mu)^2}{2\sigma^2}}(\mu-x) \\
	 & \bullet f'_x(x) \geq 0 \Leftrightarrow x\leq\mu                                         \\
	 & \bullet f''_x(x) = \left(\frac{x-\mu}{\sigma}\right)^2 - 1                              \\
	 & \bullet f''_x(x) \geq 0 \Leftrightarrow x\geq \mu+\sigma \lor x\leq \mu-\sigma
\end{align*}
Modificare $\mu$ significa ovviamente traslare la curva parallelamente all'asse delle $x$. Aumentare il valore di $\sigma$ significa diminuire l'ordinata del massimo e, conseguentemente, \qt{allargare la campana} (in quanto l'area totale sottesa deve rimanere invariata). Vale ovviamente il viceversa per una diminuzione.

Si può dimostrare che l'area sottesa alla funzione di densità è $1$:
\begin{equation*}
	\int_{-\infty}^{+\infty} \frac{1}{\sqrt{2\pi}~\sigma} e^{-\frac{1}{2}\left(\frac{x-\mu}{\sigma}\right)^2} dx = 1
\end{equation*}


\subsubsection{Funzione di ripartizione}
\begin{equation*}
	F_X(x) = \int_{-\infty}^x \frac{1}{\sqrt{2\pi}~\sigma} e^{-\frac{1}{2}\left(\frac{y-\mu}{\sigma}\right)^2} dy
\end{equation*}


\subsubsection{Valore atteso}
\begin{equation*}
	\ev{X} = \mu
\end{equation*}


\subsubsection{Varianza}
\begin{equation*}
	\var(X) = \sigma^2
\end{equation*}


\subsubsection{Distribuzione normale standard}
A partire da una variabile gaussiana $X$ si può costruire la variabile $Z$ come standardizzazione (normalizzazione) di $X$:
\begin{equation*}
	Z = \frac{x-\mu}{\sigma}
\end{equation*}

\noindent
Si verificano i seguenti risultati
\begin{align*}
	\ev{Z}  & = \ev{\frac{1}{\sigma}\ev{X-\mu}}  \\
	        & = \frac{1}{\sigma}(\ev{X}-\mu) = 0 \\
	\\
	\var(Z) & = \frac{1}{\sigma^2} \var(X-\mu)   \\
	        & = \frac{1}{\sigma^2}\var(X) = 1
\end{align*}
Ergo
\begin{equation*}
	X\sim N(\mu,\sigma) \Rightarrow Z\sim N(0,1)
\end{equation*}

Le variabili normali standard si indicano solitamente con $Z$, mentre le relative funzioni di densità e di ripartizione di indicano rispettivamente con $\phi(z)$ e $\Phi(z)$.


\subsubsection{Risultati notevoli}

\paragraph{Trasformazioni lineari} Trasformando linearmente la variabile aleatoria $X\sim N(\mu,\sigma)$, si ottiene una variabile aleatoria gaussiana $Y$:
\begin{equation*}
	X\sim N(\mu,\sigma) \Rightarrow Y\sim N(a\mu+b,a\sigma)
\end{equation*}

\paragraph{Riproducibilità} Date variabili $X_1,\dots,X_n$ gaussiane indipendenti tali che $\forall i X_i\sim N(\mu_i,\sigma_i)$:
\begin{equation*}
	Y\sim N\left(\sum_{i=1}^n \mu_i,\sqrt{\sum_{i=1}^n \sigma_i^2}\right)
\end{equation*}
Anche il modello binomiale e l'ipergeometrico, ad esempio, godono della proprietà di riproducibilità.

\paragraph{Funzione di ripartizione normale e standard} è possibile ricavare la funzione di ripartizione di una variabile aleatoria gaussiana qualsiasi conoscendo la funzione di ripartizione di una variabile standard:
\begin{align*}
	F_X(x) & = P(X\leq x)                                                 \\
	       & = P\left(\frac{X-\mu}{\sigma}\leq\frac{x-\mu}{\sigma}\right) \\
	       & = P\left(Z\leq \frac{x-\mu}{\sigma}\right)                   \\
	       & = F_Z\left(\frac{x-\mu}{\sigma}\right)                       \\
	       & = \Phi\left(\frac{x-\mu}{\sigma}\right)
\end{align*}


\subsection{Risultati notevoli sui modelli}

\subsection{Indici di variabili aleatorie}
\begin{defin}
	Data una variabile aleatoria $X$, la mediana di $X$ è un numero $m\in\R$ tale che $P(X\leq m) = P(X>m) = 1/2$.
\end{defin}

\begin{defin}
	Data una variabile aleatoria $X$, la moda di $X$ è la specificazione di densità (o massa di probabilità) massima.
\end{defin}

% TODO: questa definizione va sistemata: una specificazione come quella descritta non è unica: come ci si comporta?
\begin{defin}
	Data una variabile aleatoria $X$, il quantile di livello $q\in[0,1]$ di $X$ è la specificazione $x_q\in\R$ tale che $P(X\leq x_q) = q$.
\end{defin}


\subsection{Teorema centrale del limite}
\begin{teor}
	Siano $X_1,\dots,X_n$ variabili aleatorie indipendenti e identicamente distribuite, ossia tali che $\forall i\quad\ev{X_i}=\mu\land\var(X_i)=\sigma^2$. Allora per $n$ grandi le variabili sono distribuite in modo approssimativamente\footnote{Il simbolo $\modsim$ indica l'appartenenza approssimativa a un modello.} normale:
	\begin{gather*}
		\sum_{i=1}^n X_i \modsim N(n\mu,\sqrt{n}\sigma) \\
	\end{gather*}
	O, standardizzando
	\begin{equation*}
		\frac{\sum_{i=1}^n X_i-n\mu}{\sqrt{n}\sigma} \modsim N(0,1)
	\end{equation*}
	Ovverosia:
	\begin{equation*}
		\lim_{n\to+\infty} P\left(\frac{\sum_{i=1}^n X_i-n\mu}{\sqrt{n}\sigma}\leq x\right) = \Phi(x)
	\end{equation*}
\end{teor}

\subsubsection{Corollario}
Per il teorema centrale del limite, variabili aleatorie bernoulliane di parametri alti possono essere approssimate con il modello normale:
\begin{gather*}
	X\sim B(n,p) \\
	X = \sum_{i=1}^n X_i \modsim N(np, \sqrt{np(1-p)}) \\
	\frac{x-np}{\sqrt{np(1-p)})} \modsim N(0,1)
\end{gather*}

\subsubsection{Funzione cumulativa empirica}
La funzione cumulativa empirica dà una misura del numero di osservazioni che superano un dato input:
\begin{equation*}
	\hat F(x) = \frac{1}{n} \sum_{i=1}^n I_{(-\infty,x]}(x_i)
\end{equation*}
% TODO: check
Fatta una selezione di osservazioni sul campione, a patto che tale selezione sia coerente con la funzione di densità/massa, la funzione cumulativa empirica è un'approssimazione tanto più buona della funzione di ripartizione della selezione quanto grande è la selezione sul campione.
\section{Statistica inferenziale}
La statistica inferenziale tenta di applicare un'inferenza attraverso l'induzione. La statistica inferenziale riprende concetti della statistica descrittiva ma li reinterpreta in senso probabilistico applicando l'induzione.

\subsection{Definizioni}
Gli attori fondamentali della statistica inferenziale sono la popolazione, il campione la statistica o stimatore\footnote{in alcuni contesti, i termini statistica e stimatore differiscono leggermente di significato. Tuttavia in questo testo verranno usati equivalentemente.}, e la stima.

\begin{itemize}
	\item La popolazione è descritta come una variabile aleatoria $X$. La branca della statistica inferenziale in cui la distribuzione di $X$ è nota a meno di uno o più parametri $\theta$ si dice parametrica: $X\sim F(\theta)$. Se anche il modello della distribuzione è ignoto, si tratta di statistica inferenziale non parametrica. Nel resto di questo testo si studierà la statistica inferenziale parametrica;
	\item Poiché non sempre si vuole ricavare o approssimare il parametro di distribuzione ignoto $\theta$, ma talvolta un valore che ne deriva, si introduce la funzione $\tau(\theta)$ a indicare tale valore;
	\item L'informazione conosciuta sulla popolazione è data da un campione, talvolta espresso come una serie di variabili aleatorie $X_1,\dots,X_n$ indipendenti e identicamente distribuite, talvolta come una tupla di loro specificazioni $x_1,\dots,x_n$;
	\item L'operazione di stimare $\tau(\theta)$ consiste nella statistica o stimatore, che è una funzione $t$ che associa a una tupla possibili specificazioni del campione a un numero reale che stima il valore cercato: $t:D_X^n\to\R$. Essendo la statistica una variabile aleatoria, la si rappresenta talvolta (quando si vuole evidenziare tale lettura) con $T$;
	\item Una stima $\hat\tau$ è un'immagine della statistica e approssima $\tau(\theta)$: $\hat\tau = t(x_1,\dots,x_n)$ con istanze $X_1=x_1,\dots,X_n=x_n$.
\end{itemize}

\subsubsection{Stimatori non deviati}
\begin{defin}[stimatore non deviato]
	Uno stimatore $t$ è non deviato per una quantità $\tau(\theta)$ quando il suo valore atteso è uguale a $\tau(\theta)$:
	\begin{equation*}
		\ev{t(X_1,\dots,X_n)} = \tau(\theta)
	\end{equation*}
\end{defin}

\begin{examp}
	Uno stimatore non deviato per stimare il valore atteso, indipendentemente dalla distribuzione, è quello che fa corrispondere alle specificazioni date la loro media:
	\begin{equation*}
		t(x_1,\dots,x_n) = \frac{1}{n} \sum_{i=1}^n x_i \qquad \tau(\theta) = \ev{X}
	\end{equation*}
	Infatti:
	\begin{align*}
		\ev{t} & = \ev{\frac{1}{n} \sum_{i=1}^n X_i} \\
		       & = \frac{1}{n}\sum_{i=1}^n \ev{X_i}  \\
		       & = \frac{1}{n} \cdot n\ev{X_i}       \\
		       & = \ev{X}
	\end{align*}
	Inoltre, calcolando la varianza dello stimatore ci si accorge che essa tende a 0 per campioni molto grandi:
	\begin{equation*}
		\var\left(\frac{1}{n} \sum_{i=1}^n X_i\right) = \frac{1}{n^2} \sum_{i=1}^n \var(X_i) = \frac{n}{n^2}\var(X) = \frac{\var(X)}{n}
	\end{equation*}
\end{examp}


\subsection{Errore quadratico medio}
Il errore quadratico medio (Mean Square Error o MSE) è un modo di valutare uno stimatore di una quantità ignota:
\begin{defin}[errore quadratico medio]
	\begin{equation*}
		\MSE_{\tau(\theta)}(T) = \ev{(T-\tau(\theta))^2}
	\end{equation*}
	con $T=t(X_1,\dots,X_n)$.
\end{defin}

\begin{defin}
	Il bias di uno stimatore $T$ su $\tau(\theta)$ è la differenza tra il suo valore atteso e $\tau(\theta)$:
	\begin{equation*}
		b_{T(\theta)}(T) = \ev{T} - \tau(\theta)
	\end{equation*}
\end{defin}

In virtù delle definizioni di MSE e bias si può trarre un interessante risultato:
\begin{align*}
	\MSE_{\tau(\theta)}(T) & = \ev{(T-\tau(\theta))^2}                                                                            \\
	                       & = \ev{(T-\ev{T}+\ev{T}-\tau(\theta))^2}                                                              \\
	                       & = \ev{(T-\ev{T})^2+2(T-\ev{T})(\ev{T}-\tau(\theta))+(\ev{T}-\tau(\theta))^2}                         \\
	                       & = \ev{(T-\ev{T})^2} + 2(\ev{T}-\tau(\theta))\underbrace{\ev{T-\ev{T}}}_{0} + (\ev{T}-\tau(\theta))^2 \\
	                       & = \var(T) + (\ev{T}-\tau(\theta))^2                                                                  \\
	                       & = \var(T) + (b_{T(\theta)}(T))^2
\end{align*}
Per definizione, se $T$ è uno stimatore non deviato per $\tau(\theta)$ allora $b_{\tau(\theta)}(T)=0$ e quindi l'errore quadratico medio di $T$ è uguale alla sua varianza.


\subsection{Stimatori consistenti}
\begin{defin}[Stimatore consistente in media quadratica]
	Uno stimatore $T_n$\footnote{Più precisamente, si parla di una famiglia di stimatori $T_n$ i cui membri differiscono per dimensione del campione} è consistente in media quadratica per $\tau(\theta)$ se
	\begin{equation*}
		\lim_{n\to+\infty} \MSE_{\tau(\theta)}(T_n) = 0
	\end{equation*}
\end{defin}

\begin{defin}[Stimatore debolmente consistente]
	Uno stimatore $T_n=t(X_1,\dots,X_n)$ è debolmente consistente rispetto a una quantità ignota $\tau(\theta)$ se e solo se
	\begin{equation*}
		\forall\varepsilon>0 \quad \lim_{n\to+\infty} P(\tau(\theta)-\varepsilon \leq T_n \leq \tau(\theta)+\varepsilon) = 1
	\end{equation*}
\end{defin}

\begin{teor}
	Se uno stimatore $T$ è consistente in media quadratica per $\tau(\theta)$ allora $T$ è debolmente consistente per $\tau(\theta)$.
\end{teor}
\begin{proof}
	\begin{align*}
		P(-\varepsilon\leq T_n-\tau(\theta)\leq \varepsilon) & = P(\abs{T_n-\tau(\theta)}\leq\varepsilon)                                                                         \\
		                                                     & = P((T_n-\tau(\theta))^2\leq\varepsilon^2)                                                                         \\
		                                                     & = 1 - P((T_n-\tau(\theta))^2>\varepsilon^2)                                                                        \\
		                                                     & > 1-\frac{\ev{(T_n-\tau(\theta))^2}}{\varepsilon^2} \bc{\parbox{35mm}{disuguaglianza \ref{teor:markov} di Markov}} \\
		                                                     & = 1-\frac{\MSE_{\tau(\theta)}(T_n)}{\varepsilon^2} \to 1
	\end{align*}
	Essendo le probabilità limitate superiormente da $1$, per il teorema del confronto la probabilità iniziale tende a $1$.
\end{proof}


\subsection{Legge dei grandi numeri}
\begin{teor}[legge dei grandi numeri forte]
	\begin{equation*}
		P\left(\lim_{n\to+\infty}\bar X_n = \mu\right) = 1
	\end{equation*}
\end{teor}

\begin{teor}[legge dei grandi numeri debole]
	\begin{equation*}
		\forall\varepsilon > 0 \qquad \lim_{n\to+\infty} P\left(\abs{\bar X_n - \mu} > \varepsilon\right) = 0
	\end{equation*}
\end{teor}


\subsection{Problema dello scarto}
Un problema tipo che fa uso del teorema centrale del limite nella statistica inferenziale è il seguente.

Sia $\bar X_n$ la media campionaria dipendente da $n$ elementi e sia $\mu$ il suo valore atteso. Si vuole che lo scarto $\abs{\bar X_n - \mu}$ tra i due valori sia minore di una soglia $r$ con probabilità superiore a $1-\delta$ (con $\delta$ piccolo a piacere):
\begin{equation*}
	P(\abs{\bar X_n - \mu} \leq r) \geq 1-\delta \\
\end{equation*}

Normalizzando:
\begin{align*}
	P\left(\frac{\abs{\bar X_n - \mu}}{\sigma/\sqrt{n}} \leq \frac{r}{\sigma/\sqrt{n}}\right) & = P\left(\abs{\frac{\bar X_n - \mu}{\sigma/\sqrt{n}}} \leq \frac{r\sqrt{n}}{\sigma}\right)               \\
	                                                                                          & \approx P\left(\abs{Z} \leq \frac{r}{\sigma}\sqrt{n}\right)                                              \\
	                                                                                          & = P\left(-\frac{r}{\sigma}\sqrt{n}\leq Z\leq \frac{r}{\sigma}\sqrt{n}\right)                             \\
	                                                                                          & = \Phi\left(\frac{r}{\sigma}\sqrt{n}\right) - \Phi\left(- \frac{r}{\sigma}\sqrt{n}\right)                \\
	                                                                                          & = \Phi\left(\frac{r}{\sigma}\sqrt{n}\right) - \left(1 - \Phi\left(\frac{r}{\sigma}\sqrt{n}\right)\right)
\end{align*}

Ergo:
\begin{align*}
	2\Phi\left(\frac{r}{\sigma}\sqrt{n}\right) - 1 & \geq 1 - \delta                                                              \\
	\Phi\left(\frac{r}{\sigma}\sqrt{n}\right)      & \geq 1 - \frac{\delta}{2}                                                    \\
	\frac{r}{\sigma}\sqrt{n}                       & \geq \Phi^{-1}\left(1-\frac{\delta}{2}\right)                                \\
	n                                              & \geq \left(\frac{\sigma}{r}\Phi^{-1}\left(1-\frac{\delta}{2}\right)\right)^2
\end{align*}

Allo stesso modo si possono ricavare gli altri parametri del problema:
\begin{gather*}
	r = \frac{\sigma}{\sqrt{n}}\Phi^{-1}\left(1-\frac{\delta}{2}\right) \\
	\delta \geq 2\left(1-\Phi\left(\frac{r}{\sigma}\sqrt{n}\right)\right)
\end{gather*}
% TODO: verificare che $\sqrt{s^2}$ sia uno stimatore non deviato per $\sigma$
Mentre $\sigma$ tipicamente è stimato come $\sqrt{s^2}$ (dove $s$ è la deviazione standard campionaria).

Un'alternativa a questo tipo di stima si può fare applicando la disuguaglianza di Tchebishev:
% TODO: r=\varepsilon, coerenza a riguardo
% TODO: avere più uniformità tra i due metodi
\begin{align*}
	P(\abs{X-\mu}\geq r)                                                       & \leq \frac{\sigma^2}{r^2}     \\
	P(\abs{X-\mu} < r) = 1-P(\abs{X-\mu}\geq r)                                & \geq 1 - \frac{\sigma^2}{r^2} \\
	P(\abs{\bar X-\mu} < \varepsilon) \geq 1 - \frac{\sigma^2}{n\varepsilon^2} & \geq 1-\delta                 \\
	\frac{\sigma^2}{n\varepsilon^2}                                            & \leq \delta                   \\
\end{align*}
% TODO: spiegazione
\begin{gather*}
	n \geq \frac{\sigma^2}{\delta\varepsilon^2} \Rightarrow P(\abs{\bar X-\mu}<\varepsilon) \geq 1-\delta \\
	\delta\geq\frac{\sigma^2}{n\varepsilon^2} \\[1ex]
	\varepsilon \geq \sqrt{\frac{\sigma^2}{n\delta}}
\end{gather*}


\subsection{Processo di Poisson}
% TODO: intro
Sia $t$ una variabile temporale e $N(t)$ il numero di eventi (variabile aleatoria) che avvengono nell'intervallo di tempo $[0,t)$.

Al variare di $t$ si definisce una famiglia di variabili aleatorie secondo processi stocastici.

Il processo di Poisson parte dalle seguenti ipotesi:
\begin{enumerate}
	\item $N(0)=0$;
	\item istanze di $N$ sono indipendenti per intervalli disgiunti;
	\item $\displaystyle\lim_{h\to0}\frac{P(N(h)=1)}{h}=\lambda$;
	\item $\displaystyle\lim_{h\to0}\frac{P(N(h)\geq 2)}{h}=0$;
\end{enumerate}
In un contesto in cui valgono le ipotesi la variabile $N(t)$ è distribuita secondo il modello di Poisson con parametro $\lambda t$:
\begin{equation*}
	N(t)\sim P(\lambda t)
\end{equation*}

% TODO: sistemare lingua
Ipotizzato $N(t)=k\in\N$ e scelto un $n\in\N$, si divide l'intervallo $[0,t)$ in $n$ parti uguali $[\frac{m}{n}t,\frac{m+1}{n}t)$. Potendo scegliere $n$ grande a piacere si può scegliere in particolare $n>k$. Chiamato con $A$ l'evento che corrisponde a ognuno dei $k$ eventi avviene in un intervallo diverso e $B$ il suo complementare, se si verifica $B$ esiste un intervallo tale che due eventi avvengono in esso.

Poiché per l'ipotesi 4 la probabilità di un tale evento per uno specifico intervallo è tendente a $0$ ($n$ grande a piacere) e per l'ipotesi 2 la probabilità di un tale evento in qualunque intervallo è calcolabile come il prodotto delle singole, la probabilità complessiva è il prodotto di fattori nulli ed è pertanto uguale a $0$.

La probabilità che ogni intervallo contenga esattamente un evento, per l'ipotesi 3 è tendente a
\begin{align*}
	P(N(t) = k) & = P(A)                                                                                 \\
	            & = \binom{n}{k}\left(\lambda\frac{t}{n}\right)^k\left(1-\lambda\frac{t}{n}\right)^{n-k} \\
	            & \sim B\left(n,\frac{\lambda t}{n}\right)
\end{align*}
Poiché il prodotto dei parametri della binomiale trovata è costante, la distribuzione è ben descritta da un modello di Poisson di parametro $\lambda t$:
\begin{equation*}
	N(t)\sim P(\lambda t)
\end{equation*}
Da cui
\begin{equation*}
	P(N(t)=k) = e^{-\lambda t}\frac{\dots}{\dots}\dots
\end{equation*}


\end{document}
