%% Copyright (C) 2021 Alessandro Clerici Lorenzini
%
% This work may be distributed and/or modified under the
% conditions of the LaTeX Project Public License, either version 1.3
% of this license or (at your option) any later version.
% The latest version of this license is in
%   http://www.latex-project.org/lppl.txt
% and version 1.3 or later is part of all distributions of LaTeX
% version 2005/12/01 or later.
%
% This work has the LPPL maintenance status `maintained'.
%
% The Current Maintainer of this work is Alessandro Clerici Lorenzini
%
% This work consists of the files listed in work.txt


% TODO: valutare le funzioni (ripartizione e massa) nel dominio negativo
\section{Modelli}
I modelli di distribuzione consistono in risultati notevoli corrispondenti a variabili aleatorie che rispettano determinate definizioni ricorrenti.

% TODO: bisognerebbe aggiungere anche il supporto, tuttavia non riesco a sistemare la geometria della pagina in un modo soddisfacente che non causi warning.
% TODO: per quelli non visti, è sufficiente fare le somme delle funzioni di massa. Per il problema di spazio già citato, per ora sono omesse
% TODO: controllare la mancanza di funzioni indicatrici
\begin{sidewaystable}
	\centering
	\begin{tabular}{llllll}
		\toprule
		\bfseries Modello           & \bfseries Parametri   & \bfseries F. di massa/densità                                    & \bfseries F. di ripartizione                                                                                          & \bfseries V. atteso & \bfseries Varianza    \\
		\midrule
		\bfseries Bernoulli         & $X\sim B(p)$          & $p^x(1-p)^{1-x}I_{\{0,1\}}(x)$                                   & $\begin{cases}0\quad & x<0\\1-p\quad & 0\leq x<1\\1\quad & x\leq 1\end{cases}$                                                                                           & $p$                 & $p(1-p)$              \\[5ex]
		\bfseries Binomiale         & $X\sim B(n,p)$        & $\displaystyle\binom{n}{i} p^i(1-p)^{n-i}$                       & $\begin{cases}\sum_{i=0}^{\floor{x}}\binom{n}{i} p^i (1-p)^{n-i}\quad & x\leq n \\1 & x>n \end{cases}$                                                                                           & $np$                & $np(1-p)$             \\[3ex]
		\bfseries Uniforme discreto & $X\sim U(n)$          & $\dfrac{1}{n} I_{\{1,\dots,n\}}(i)$                              & $\dfrac{\floor{x}}{n}\cdot I_{\{1,\dots,n\}}+I_{\{n,\dots,+\infty\}}$                                                 & $\dfrac{n+1}{2}$    & $\dfrac{n^2-1}{12}$   \\[2ex]
		\bfseries Geometrico        & $X\sim G(p)$          & $(1-p)^ip ~ I_\N (i)$                                            & $1 - (1-p)^{\floor{x}+1}  $                                                                                           & $ \dfrac{1-p}{p}$   & $\dfrac{1-p}{p^2}$    \\[1ex]
		\bfseries Poisson           & $X\sim P(\lambda)$    & $e^{-\lambda}\frac{\lambda^i}{i!} ~ I_\N(i)$                     & [non visto]                                                                                                           & $\lambda$           & $\lambda$             \\[2ex]
		\bfseries Ipergeometrico    & $X\sim ?(N,M,n)$      & $\dfrac{\binom{N}{i}\binom{M}{n-i}}{\binom{N+M}{n}}$             & [non visto]                                                                                                           & $np$                & $\dfrac{NM}{(N+M)^2}$ \\[2ex]
		\bfseries Uniforme continuo & $X\sim U([a,b])$      & $\dfrac{1}{b-a} I[a,b](x)$                                       & $ \dfrac{x-a}{b-a}\cdot I_{[a,b]}(x)+I_{(b,+\infty)}(x)$                                                              & $\dfrac{b+a}{2}$    & $\dfrac{(b-a)^2}{12}$ \\[3ex]
		\bfseries Esponenziale      & $X\sim E(\lambda)$    & $\lambda e^{-\lambda x} I_{\R^+}(x)$                             & $1-e^{-\lambda x}$                                                                                                    & $\frac{1}{\lambda}$ & $\frac{1}{\lambda^2}$ \\[1ex]
		\bfseries Gaussiano         & $X\sim N(\mu,\sigma)$ & $\dfrac{1}{\sqrt{2\pi}~\sigma}e^{-\dfrac{(x-\mu)^2}{2\sigma^2}}$ & $\displaystyle\int_{-\infty}^x \dfrac{1}{\sqrt{2\pi}~\sigma} e^{-\frac{1}{2}\left(\dfrac{y-\mu}{\sigma}\right)^2} dy$ & $\mu$               & $\sigma^2$            \\
		\bottomrule
	\end{tabular}
	\caption{Tabella riassuntiva dei modelli di distribuzione.}
\end{sidewaystable}

\subsection{Modello bernoulliano}
Il modello bernoulliano impone alle sue variabili aleatorie di avere specificazione binaria, cioè $D_X=\{0,1\}$ per qualunque variabile $X$.

Una variabile aleatoria bernoulliana di probabilità di successo ($X=1$) $p$ si indica con
\begin{equation*}
	X\sim B(p)
\end{equation*}


\subsubsection{Funzione di massa di probabilità}
La funzione di massa di probabilità di una variabile aleatoria bernoulliana è
\begin{equation*}
	p_X(x)=P(X=x)=p^x(1-p)^{1-x}I_{\{0,1\}}(x)=\begin{cases}
		1-p \qquad & x=0               \\
		p \qquad   & x=1               \\
		0 \qquad   & \text{altrimenti}
	\end{cases}
\end{equation*}


\subsubsection{Valore atteso}
In analogia con la proprietà \ref{prop:indvalatt}, il valore atteso di una variabile aleatoria bernoulliana è $p$:
\begin{equation*}
	\ev{X} = p
\end{equation*}
E ovviamente
\begin{equation*}
	\ev{X^2}=\ev{X}
\end{equation*}


\subsubsection{Varianza}
In analogia con la proprietà \ref{prop:indvar}, la varianza di una variabile aleatoria bernoulliana è $p(1-p)$:
\begin{equation*}
	\var(X) = p(1-p)
\end{equation*}


\subsubsection{Funzione di ripartizione}
La funzione di ripartizione per variabili bernoulliane è ovviamente uguale a $0$ per $x<0$, $1-p$ per $0\leq x<1$ e $1$ per $x\geq 1$.



\subsection{Modello binomiale}
Il modello binomiale consiste in $n$ ripetizioni di un esperimento bernoulliano di probabilità $p$. Una variabile aleatoria binomiale corrisponde al numero di successi tra gli $n$ esperimenti.
\begin{equation*}
	X \sim B(n, p)\qquad D_X=\{0,1,\dots,n\}
\end{equation*}


\subsubsection{Funzione di massa di probabilità}
Per definizione:
\begin{equation*}
	p_X(i) = P(X=i)
\end{equation*}
Tale probabilità è l'intersezione degli eventi indipendenti che consistono nel successo dal primo all'$i$-esimo e insuccesso dall'$i+1$-esimo all'$n$-esimo. In quanto probabilità di eventi indipendenti, essa può essere espressa come il prodotto delle singole probabilità. Ognuna delle probabilità di successo, per come è costruito il modello binomiale, è $p$, mentre ognuna delle probabilità di insuccesso è $1-p$. Ogni combinazione in cui $i$ esperimenti hanno successo e $n-i$ falliscono è valida, perciò:
\begin{equation}
	P(X=i) = \binom{n}{i} p^i(1-p)^{n-i}
\end{equation}

Concorde con la \eqref{eq:sommamassa}, la somma delle immagini della funzione di massa di probabilità è 1:
\begin{equation*}
	\sum_{i=1}^n p_X(i) = \sum_{i=1}^n \binom{n}{i} p^i(1-p)^{n-i} = (p+1-p)^n = 1
\end{equation*}


\subsubsection{Valore atteso}
Essendo ogni variabile aleatoria binomiale la somma di variabili aleatorie bernoulliane:
\begin{equation}
	\ev{X} = \ev{\sum_{i=0}^n X_i} = \sum_{i=0}^n \ev{X_i} = \sum_{i=0}^n p = np
\end{equation}


\subsubsection{Varianza}
Essendo le componenti bernoulliane indipendenti:
\begin{equation}
	\var(X)=\sum_{i=1}^n \var(X_i) = \sum_{i=1}^n p(1-p) = np(1-p)
\end{equation}


\subsubsection{Funzione di ripartizione}
Per calcolare con un'unica formula la funzione di ripartizione si aggiungono due funzioni indicatrici, che agiscono se $x>n$:
\begin{align}
	F_X(x) & = P(X\leq x)                                                                           \nonumber            \\
	       & = I_{(n,+\infty)}(x) + I_{[0,n]}(x) \sum_{i=0}^{\floor{x}}\binom{n}{i} p^i (1-p)^{n-i} \label{eq:ripabinom} \\
	       & = \begin{cases}
		\sum_{i=0}^{\floor{x}}\binom{n}{i} p^i (1-p)^{n-i}\quad & x\leq n \\
		1                                                       & x>n     \\
	\end{cases} \nonumber
\end{align}

\subsubsection{Relazioni tra variabili binomiali}
Siano $X_1$ e $X_2$ due variabili aleatorie definite su modelli binomiali che differiscono solo per il numero di esperimenti:
\begin{align*}
	 & X_1\sim B(n, p)\quad & X_1=\sum_{i=1}^n X_{1,i}\qquad & X_{1,i}\sim B(p)~\forall i\in\{1,\dots,n\} \\
	 & X_2\sim B(m, p)\quad & X_1=\sum_{j=1}^m X_{2,j}\qquad & X_{2,j}\sim B(p)~\forall j\in\{1,\dots,m\}
\end{align*}

\noindent
Se $X_1$ e $X_2$ sono indipendenti, allora:
\begin{equation*}
	X_1+X_2 = \sum_{i=1}^n X_{1,i} + \sum_{j=1}^m X_{2,j} = \sum_{i=1}^{n+m} Y_i = Y
\end{equation*}
dove $Y\sim B(n+m, p)$.



\subsection{Modello uniforme discreto} \label{subsec:unifdisc}
Nel modello uniforme discreto le variabili aleatorie consistono nell'esito di un esperimento da $n$ esiti possibili equiprobabili:
\begin{equation*}
	X\sim U(n)
\end{equation*}


\subsubsection{Funzione di massa di probabilità}
Essendo gli $n$ esiti equiprobabili, la funzione di massa di probabilità assume il valore $\frac{1}{n}$ per tutti i valori dell'input compresi tra gli esiti (numerati qui da $1$ a $n$):
\begin{equation*}
	p_X(i) = P(X=i) = \frac{1}{n} I_{\{1,\dots,n\}}(i)
\end{equation*}


\subsubsection{Funzione di ripartizione}
\begin{equation*}
	\forall x\leq n\qquad F_X(x) = P(X\leq x) = \sum_{i=1}^{\floor{x}} P(X=i) = \sum{i=1}^{\floor{x}} \frac{1}{n} = \frac{\floor{x}}{n}
\end{equation*}
Come per la \eqref{eq:ripabinom}, si aggiungono funzioni indicatrici per regolare il valore oltre $n$:
\begin{equation}
	F_X(x)=\frac{\floor{x}}{n} I_{[1,n]}(x)+I_{(n,+\infty)}(x)
\end{equation}


\subsubsection{Valore atteso}
Banalmente, usando la definizione:
\begin{equation} \label{eq:disunvalat}
	\ev{X} = \sum_{i=1}^n i P(X=i) = \frac{1}{n} \sum_{i=1}^n i = \frac{1}{n}\frac{n(n+1)}{2} = \frac{n+1}{2}
\end{equation}


\subsubsection{Varianza}
Usando la definizione equivalente di varianza e quanto appena calcolato:
\begin{align*}
	\var(X) & = \ev{X^2} - \ev{X}^2 =                                       \bc{proprietà \ref{prop:varalt}}           \\
	        & = \sum_{i=1}^n i^2 P(X=i) - \left(\frac{n+1}{2}\right)^2      \bc{valore atteso e \eqref{eq:disunvalat}} \\
	        & = \frac{1}{n} \sum_{i=1}^n i^2 - \left(\frac{n+1}{2}\right)^2 \bc{ipotesi di equiprobabilità}            \\
	        & = \frac{(n+1)(2n+1)}{6}-\left(\frac{n+1}{2}\right)^2          \bc{somma notevole}                        \\
	        & = (n+1)\left(\frac{2n+1}{6}+\frac{n+1}{4}\right)              \bc{raccogliendo $n+1$}                    \\
	        & = (n+1)\left(\frac{n-1}{12}\right)                                                                       \\
	        & = \frac{n^2-1}{12}
\end{align*}



\subsection{Modello geometrico}
Una variabile aleatoria geometrica corrisponde al numero di insuccessi prima del primo successo in una sequenza di esperimenti bernoulliani con lo stesso parametro $p$ e tra loro indipendenti.

Per $p=0$ non si ottiene mai un successo, pertanto la variabile geometrica non è definita. Per $p=1$ la variabile assume necessariamente il valore $0$.

Il supporto di una variabile geometrica è l'insieme dei naturali.

\begin{equation*}
	X\sim G(p)\qquad D_X=\N
\end{equation*}


\subsubsection{Funzione di massa di probabilità}
La funzione di massa di probabilità in $i$ è uguale alla probabilità di insuccesso per ognuna delle $i$ ripetizioni indipendenti per la probabilità di successo della $i+1$-esima.

Come sempre si aggiunge una funzione indicatrice per aggiustare il dominio di $F_X$:
\begin{equation} \label{eq:geommasprob}
	F_X(i) = P(X=i) = (1-p)^ip ~ I_\N (i)
\end{equation}

La somma delle immagini della funzione di massa di probabilità converge, come dovrebbe, a $1$:
\begin{align*}
	\sum_{i=0}^{+\infty} P(X=i) & = \sum_{i=0}^{+\infty} p(1-p)^i \\
	                            & = p\sum_{i=0}^{+\infty} (1-p)^i \\
	                            & = p\frac{1}{1-(1-p)}            \\
	                            & = 1
\end{align*}


\subsubsection{Valore atteso}
Tramite la definizione:
\begin{align*}
	\ev{X} & = \sum_{i=0}^{+\infty} i P(X=i)                    \bc{definizione \ref{def:valatt}}         \\
	       & = \sum_{i=0}^{+\infty} ip(1-p)^i                   \bc{\eqref{eq:geommasprob}}               \\
	       & = p(1-p)\sum_{i=0}^{+\infty} i(1-p)^{i-1}                                                    \\
	       & = -p(1-p) \frac{d}{dx} \sum_{i=0}^{+\infty} (1-p)^i \bc{derivata di $(1-p)^i$ e della somma} \\
	       & = -p(1-p) \frac{d}{dx} \frac{1}{p}                  \bc{serie geometrica di ragione $1-p$}   \\
	       & = \frac{p(1-p)}{p^2} = \frac{1-p}{p}                 \bc{derivando}
\end{align*}


\subsubsection{Varianza}
Volendo usare la forma equivalente \eqref{eq:varalt} di cui alla proprietà \ref{prop:varalt}, si calcola innanzitutto il valore atteso del quadrato della variabile:
\begin{align*}
	\ev{X^2} & = \sum_{i=0}^{+\infty} i^2 p(1-p)^i                                                                                                          \\
	         & = p(1-p) \sum_{i=0}^{+\infty} i^2 (1-p)^{i-1}                                                                                                \\
	         & = -p(1-p) \sum_{i=0}^{+\infty} i \frac{d}{dp}(1-p)^i \bc{derivata di $(1-p)^i$}                                                              \\
	         & = -p(1-p) \frac{d}{dp}\sum_{i=0}^{+\infty} i (1-p)^i \bc{\parbox{42.5mm}{prodotto di una derivata per una costante e derivata di una somma}} \\
	         & = -p(1-p) \frac{d}{dp}(1-p)\sum_{i=0}^{+\infty} i (1-p)^{i-1}                                                                                \\
	         & = p(1-p) \frac{d}{dp}(1-p) \frac{d}{dp} \sum_{i=0}^{+\infty} (1-p)^i                                                                         \\
	         & = -p(1-p) \frac{d}{dp} \frac{1-p}{p^2}                                                                                                       \\
	         & = -p(1-p) \frac{-p^2-2p(1-p)}{p^4}                                                                                                           \\
	         & = (1-p) \frac{p+2(1-p)}{p^2}                                                                                                                 \\
	         & = \frac{(1-p)(2-p)}{p^2}
\end{align*}
Da cui:
\begin{align*}
	\var(X) & = \ev{X^2}-\ev{X}^2                                     \\
	        & = \frac{(1-p)(2-p)}{p^2} - \left(\frac{1-p}{p}\right)^2 \\
	        & = \frac{(1-p)((2-p)-(1-p))}{p^2}                        \\
	        & = \frac{1-p}{p^2}
\end{align*}


\subsubsection{Funzione di ripartizione}
\begin{align*}
	F_X(n) & = P(X\leq n) = 1-P(X>n) =                                                                                                 \\
	       & = 1 - \sum_{i=n+1}^{+\infty} P(X=i)                                                                                       \\
	       & = 1 - \sum_{i=n+1}^{+\infty} p(1-p)^i                                                                                     \\
	       & = 1 - p(1-p)^{n+1} \sum_{i=n+1}^{+\infty} (1-p)^{i-(n+1)} \justif{~}{moltiplicando per $\frac{(1-p)^{n+1}}{(1-p)^{n+1}}$} \\
	       & = 1 - p(1-p)^{n+1} \sum_{i=0}^{+\infty} (1-p)^i \justif{~}{sostituzione: $i=i-(n+1)$}                                     \\
	       & = 1 - p(1-p)^{n+1} \frac{1}{1-(1-p)}                                                                                      \\
	       & = 1 - (1-p)^{n+1}                                                                                                         \\
\end{align*}
Questo risultato è in realtà banale se si applica il concetto semantico alla variabile geometrica.

\begin{equation*}
	F_X(x) = P(X\leq x) = 1-P(X>x) = 1-(1-p)^{\floor{x}+1}
\end{equation*}

\subsubsection{Assenza di memoria} \label{geom-assmem}
Come si può intuire, la probabilità di costante insuccesso all'$i+j$-esimo esperimento non è condizionata dalla probabilità di costante insuccesso all'$i$-esimo. Questo risultato prende il nome di assenza di memoria.
\begin{align*}
	P(X\geq i+j \mid X\geq i) & = \frac{P(X\geq i+j \cap X\geq i)}{P(X\geq i)} \\
	                          & = \frac{P(X\geq i+j)}{P(X\geq i)}              \\
	                          & = \frac{(1-p)^{i+j}}{(1-p)^i}                  \\
	                          & = (1-p)^j                                      \\
	                          & = P(X\geq j)
\end{align*}



\subsection{Modello di Poisson}
\begin{equation*}
	X\sim P(\lambda) \qquad D_X = \N \qquad \lambda>0
\end{equation*}

\subsubsection{Funzione di massa di probabilità}
\begin{equation*}
	P_X(i) = P(X=i) = e^{-\lambda}\frac{\lambda^i}{i!} ~ I_\N(i)
\end{equation*}

Ancora una volta la somma delle immagini della funzione di massa di probabilità converge a $1$:
\begin{equation} \label{eq:poisummas}
	\sum_{i=0}^{+\infty} e^{-\lambda}\frac{\lambda^i}{i!} = e^{-\lambda}\sum_{i=0}^{+\infty} \frac{\lambda^i}{i!} = e^{-\lambda}e^\lambda = 1
\end{equation}


\subsubsection{Valore atteso}
Tramite la definizione:
\begin{align}
	\ev{X} & = \sum_{i=0}^{+\infty} i P(X=i) = \sum_{i=1}^{+\infty} i P(X=i) \bc{definizione \ref{def:valatt}} \nonumber \\
	       & = \sum_{i=1}^{+\infty} i e^{-\lambda}\frac{\lambda^i}{i!} \label{eq:poinotevolelambda}                      \\
	       & = \lambda e^{-\lambda} \sum_{i=1}^{+\infty} \frac{\lambda^{i-1}}{(i-1)!} \nonumber                          \\
	       & = \lambda e^{-\lambda} \sum_{i=0}^{+\infty}\frac{\lambda^i}{i!} \bc{sostituzione: $i=i-1$} \nonumber        \\
	       & = \lambda e^{-\lambda} e^\lambda \nonumber                                                                  \\
	       & = \lambda
\end{align}


\subsubsection{Varianza}
Volendo usare la forma equivalente \eqref{eq:varalt}, si calcola innanzitutto il valore atteso di $X^2$:
\begin{align*}
	\ev{X^2} & = \sum_{i=1}^{+\infty} i^2 e^{-\lambda}\frac{\lambda^i}{i!}                                                                                                                              \\
	         & = \sum_{i=1}^{+\infty} i e^{-\lambda} \frac{\lambda^i}{(i-1)!}                                                                                                                           \\
	         & = \lambda\sum_{i=1}^{+\infty} i e^{-\lambda}\frac{\lambda^{i-1}}{(i-1)!}                                                                                                                 \\
	         & = \lambda\sum_{i=1}^{+\infty} (i-1+1) e^{-\lambda}\frac{\lambda^{i-1}}{(i-1)!}                                                                                                           \\
	         & = \lambda\sum_{i=1}^{+\infty}\left((i-1)e^{-\lambda}\frac{\lambda^{i-1}}{(i-1)!}+e^{-\lambda}\frac{\lambda^{i-1}}{(i-1)!}\right)                                                         \\
	         & = \lambda\sum_{i=1}^{+\infty} (i-1)e^{-\lambda}\frac{\lambda^{i-1}}{(i-1)!} + \lambda\sum_{i=1}^{+\infty} + e^{-\lambda}\frac{\lambda^{i-1}}{(i-1)!}                                     \\
	         & = \lambda\underbrace{\sum_{i=0}^{+\infty} ie^{-\lambda}\frac{\lambda^i}{i!}}_{\lambda} + \lambda \underbrace{\sum_{i=0}^{+\infty} e^{-\lambda}\frac{\lambda^i}{i!}}_{1} \bc{con $i=i-1$} \\
	         & = \lambda^2 + \lambda \bc{\eqref{eq:poinotevolelambda} e \eqref{eq:poisummas}}
\end{align*}
Ergo
\begin{align}
	\var(X) & = \ev{X^2} - \ev{X}^2         \nonumber \\
	        & = \lambda^2+\lambda-\lambda^2 \nonumber \\
	        & = \lambda
\end{align}

\subsubsection{Approssimazione del modello binomiale} \label{subsub:binompois}
Il modello di Poisson è strettamente legato al modello binomiale. Infatti, se il prodotto dei parametri di una binomiale è costante, per $n$ grandi essa è ben approssimata da una variabile di Poisson che ha come parametro tale prodotto:
\begin{equation*}
	X\sim B(n, p)\qquad\text{con }np=\lambda
\end{equation*}
Per $n\to+\infty$:
\begin{align*}
	P(X=i) & = \binom{n}{i} p^i (1-p)^{n-i}                                                                                                                                                                                                                                                \\
	       & = \binom{n}{i}\left(\frac{\lambda}{n}\right)^i\left(1-\frac{\lambda}{n}\right)^{n-i}                                                                                                                                                                                          \\
	       & = \frac{n(n-1) \dots (n-i+1)}{i!} \cdot \frac{\lambda^i}{n^i} \left( 1 - \frac{\lambda}{n} \right)^{n-i}                                                                                                                                                                      \\
	       & = \frac{n(n-1) \dots (n-i+1)}{n^i} \cdot \frac{\lambda^i}{i!} \left( 1 - \frac{\lambda}{n} \right)^{n-i}                                                                                                                                                                      \\
	       & = \underbrace{\frac{n}{n}}_{\to 1} \cdot \underbrace{\frac{n-1}{n}}_{\to 1} \dots \underbrace{\frac{n-i+1}{n}}_{\to 1} \cdot \frac{\lambda^i}{i!} \cdot \underbrace{\frac{\left( 1 - \frac{\lambda}{n} \right)^n}{\left( 1 - \frac{\lambda}{n} \right)^i}}_{\to e^{-\lambda}} \\
	       & \to \frac{\lambda^i}{i!} e^{-\lambda}
\end{align*}



\subsection{Modello ipergeometrico}
Il modello ipergeometrico descrive il classico problema dell'urna. Dati $N$ oggetti funzionanti e $M$ oggetti difettosi, sia $n$ il numero di estrazioni senza reimmissione. La variabile aleatoria ipergeometrica X è il numero di oggetti funzionanti nelle $n$ estrazioni.
\begin{equation*}
	X\sim ?(?)
\end{equation*}

Il modello è valido solo se $P(X=0)=0$.


\subsubsection{Funzione di massa di probabilità}
Il numero di casi possibili sono le combinazioni di $n$ estrazioni senza reimmissione da un gruppo di $N+M$. Il numero di casi favorevoli si può calcolare applicando il principio fondamentale del calcolo combinatorio.
\begin{equation*}
	P(X=i) = \frac{\binom{N}{i}\binom{M}{n-i}}{\binom{N+M}{n}}
\end{equation*}


\subsubsection{Valore atteso}
Al fine di calcolare il valore atteso si sfrutta un approccio decomposizionale: si introducono $n$ variabili aleatorie $X_i$, ognuna legata a un'estrazione, tali che
\begin{equation*}
	X_i = \begin{cases}
		1 & \text{l'$i$-esimo oggetto estratto funziona} \\
		0 & \text{altrimenti}
	\end{cases}
\end{equation*}

Per tali variabili vale
\begin{equation*}
	P(X_i=1) = \frac{N}{N+M} =: p = \ev{X_i}
\end{equation*}

Da cui
\begin{align*}
	\ev{X} & = \ev{\sum_{i=1}^n X_i} \\
	       & = \sum_{i=1}^n \ev{X_i} \\
	       & = np
\end{align*}


\subsubsection{Varianza}
Applicando la proprietà \ref{prop:varalt} si calcola la varianza delle singole $X_i$:
\begin{align*}
	\var(X_i) & = \ev{X_i^2}-\ev{X_i}^2                                   \\
	          & = \ev{X_i}(1-\ev{X_i})          \bc{idempotenza di $X_i$} \\
	          & = \frac{N}{N+M} + \frac{M}{N+M}                           \\
	          & = \frac{NM}{(N+M)^2}
\end{align*}

Essendo le variabili $X_i$ non indipendenti, la varianza della loro somma non è uguale alla somma delle varianze. Si può comunque applicare la proprietà \ref{prop:varsumcov} e passare per le covarianze:
\begin{align*}
	\cov(X_i,X_j) & = \ev{X_i X_j} - \ev{X_i}\ev{X_j}                                \\
	              & = \ev{X_i=1\cap X_j=1} - \left(\frac{N}{N+M}\right)^2            \\
	              & = P(X_j=1 \mid X_i=1)P(X_i=1) - \left(\frac{N}{N+M}\right)^2     \\
	              & = \frac{N-1}{N+M-1} \frac{N}{N+M} - \left(\frac{N}{N+M}\right)^2 \\
	              & = \frac{N}{N+M} \left(\frac{N-1}{N+M-1} - \frac{N}{N+M}\right)   \\
	              & = \frac{-NM}{(N+M-1)(N+M)^2}
\end{align*}
Ergo
\begin{align*}
	\var(X) & = \sum_{i=1}^n \var(X_i) + \sum_{i\neq j}^n \cov(X_i,X_j)     \\
	        & = n\frac{NM}{(N+M)^2} - n(n-1)\cdot\frac{-NM}{(N+M-1)(N+M)^2} \\
	        & = n\frac{NM}{(N+M)^2}\left(1-(n-1)\frac{1}{N+M-1}\right)      \\
	        & = np(1-p)\left(1-\frac{n-1}{N+M-1}\right)
\end{align*}
Per $N+M\to+\infty$ il modello si semplifica in un modello binomiale:
\begin{equation*}
	\to np(1-p)
\end{equation*}


\subsection{Modello uniforme continuo}
Il modello uniforme continuo estende al continuo i concetti visti alla sezione \ref{subsec:unifdisc} per il modello uniforme discreto e viene determinato da un intervallo equivalentemente aperto o chiuso:
\begin{equation*}
	X \sim U([a,b])
\end{equation*}


\subsubsection{Funzione di densità di probabilità}
\begin{equation*}
	f(X)(x) = \frac{1}{b-a} I[a,b](x)
\end{equation*}
essendo la densità costante, per $I\subseteq[a,b]$:
\begin{equation*}
	P(X\in I) = \frac{|I|}{b-a}
\end{equation*}
dove $|I|$ è la somma delle ampiezze $p-q$ degli intervalli disgiunti $[p,q]$ da cui $I$ è composto.

Come da definizione, integrando nell'intero $\R$ la funzione di densità di probabilità si ottiene $1$:
\begin{equation*}
	\int_{-\infty}^{+\infty}f_X(x) = \int_a^b\frac{1}{b-a}dx = \frac{1}{b-a} \eval{x}{a}{b} = 1
\end{equation*}


\subsubsection{Funzione di ripartizione}
Per definizione la funzione di ripartizione è la funzione integrale della funzione di densità:
\begin{align*}
	F_X(x) & = P(X\leq x)                   \\
	       & = \int_{-\infty}^x f_X(u)du    \\
	       & = \int_a^x \frac{1}{b-a}du     \\
	       & = \frac{1}{b-a} \eval{u}{a}{x} \\
	       & = \frac{x-a}{b-a}
\end{align*}

Come sempre funzioni indicatrici aggiustano il risultato per punti non appartenenti all'intervallo:
\begin{equation*}
	F_X(x) = \frac{x-a}{b-a} I_{[a,b]}(x) + I_{(b,+\infty)}(x)
\end{equation*}


\subsubsection{Valore atteso}
Applicando la definizione:
\begin{align*}
	\ev{X} & = \int_a^b x f_X(x)dx                      \\
	       & = \frac{1}{b-a} \int_a^b x ~ dx            \\
	       & = \frac{1}{b-a} \eval{\frac{x^2}{2}}{a}{x} \\
	       & = \frac{1}{b-a} \cdot \frac{b^2-a^2}{2}    \\
	       & = \frac{b+a}{2}
\end{align*}


\subsubsection{Varianza}
Come di consueto si intende applicare la proprietà \ref{prop:varalt}, pertanto si calcola innanzitutto il valore atteso di $X^2$:
\begin{align*}
	\ev{X^2} & = \int_a^b x^2 f_X(x)dx                    \\
	         & = \frac{1}{b-a}\int_a^b x^2 ~ dx           \\
	         & = \frac{1}{b-a} \eval{\frac{x^3}{3}}{a}{b} \\
	         & = \frac{b^3-a^3}{3(b-a)}                   \\
	         & = \frac{a^2+ab+b^2}{3}
\end{align*}

E infine:
\begin{align*}
	\var(X) & = \ev{X^2} - \ev{X}^2                      \\
	        & = \frac{a^2+ab+b^2}{3} - \frac{(a+b)^2}{4} \\
	        & = \frac{(b-a)^2}{12}
\end{align*}


\subsection{Modello esponenziale}
Nel modello esponenziale la funzione di densità è esponenziale.
\begin{equation*}
	X\sim E(\lambda) \qquad \lambda\in\R^+ \quad D_X=\R^+
\end{equation*}


\subsubsection{Funzione di densità di probabilità}
\begin{equation*}
	f_X(x) = \lambda e^{-\lambda x} I_{\R^+}(x)
\end{equation*}

Il modello esponenziale si usa per modellare il tempo che intercorre tra due eventi.

La funzione di densità rispetta la definizione, infatti:
\begin{align*}
	\int_0^{+\infty} f_X(x)dx & = \int_0^{+\infty} \lambda e^{-\lambda x} dx       \\
	                          & = \int_0^{+\infty} e^{-y}dy \bc{con $y=\lambda x$} \\
	                          & = \eval{-e^{-y}}{0}{+\infty}                       \\
	                          & = 0 + e^{-0}                                       \\
	                          & = 1
\end{align*}


\subsubsection{Funzione di ripartizione}
Applicando la definizione di funzione di ripartizione continua:
\begin{align*}
	F_X(x) & = \int_0^x f_X(y)dy                                  \\
	       & = \int_0^x \lambda e^{\lambda y}dy                   \\
	       & = \int_0^{\lambda x} e^{-z}dz \bc{con $z=\lambda y$} \\
	       & = \eval{-e^{-z}}{0}{\lambda x}                       \\
	       & = -e^{-\lambda x} + e^0                              \\
	       & = 1-e^{-\lambda x}
\end{align*}
Aggiungendo una funzione indicatrice:
\begin{equation}
	F_X(x) = (1 - e^{-\lambda x})I_{\R^+}(x)
\end{equation}


\subsubsection{Valore atteso}
Applicando la definizione:
\begin{align*}
	\ev{X} & = \int_0^{+\infty} x f_X(x) dx                                                             \\
	       & = \int_0^{+\infty} x\lambda e^{-\lambda x}dx                                               \\
	       & = \eval{-x e^{-\lambda x}}{0}{+\infty} + \int_0^{+\infty} e^{-\lambda x} dx \bc{per parti} \\
	       & = \int_0^{+\infty} e^{-\lambda x} dx                                                       \\
	       & = \frac{1}{\lambda} \underbrace{\int_0^{+\infty} \lambda e^{-\lambda x}}_{1}               \\
	       & = \frac{1}{\lambda} \bc{proprietà di $f_X$}
\end{align*}


\subsubsection{Varianza}
Volendo usare la forma equivalente \eqref{eq:varalt}, si calcola innanzitutto il valore atteso di $X^2$:
\begin{align*}
	\ev{X^2} & = \int_0^{+\infty} x^2\lambda e^{-\lambda x} dx                                 \\
	         & = \eval{-x^2 e^{-\lambda x}}{0}{+\infty} + \int_0^{+\infty} 2xe^{-\lambda x} dx \\
	         & = 2\int_0^{+\infty} xe^{-\lambda x}dx                                           \\
	         & = \frac{2}{\lambda} \int_0^{+\infty} \lambda xe^{-\lambda x}dx                  \\
	         & = \frac{2}{\lambda} \ev{X} = \frac{2}{\lambda^2}
\end{align*}
Da cui:
\begin{align*}
	\var(X) & = \ev{X^2} - \ev{X}^2                                           \\
	        & = \frac{2}{\lambda^2}-\frac{1}{\lambda^2} = \frac{1}{\lambda^2}
\end{align*}


\subsubsection{Assenza di memoria}
Le variabili di modello esponenziale godono della proprietà di assenza di memoria (già vista per il modello geometrico al paragrafo \ref{geom-assmem}):
\begin{equation*}
	P(X>x) = 1 - F_X(x) = e^{-\lambda x}
\end{equation*}
Quindi:
\begin{align*}
	P(X>s+t) & = e^{-\lambda (s+t)}           \\
	         & = e^{-\lambda s}e^{-\lambda t} \\
	         & = P(X>s)P(X>t)
\end{align*}
Da cui
\begin{align*}
	P(X>s) & = \frac{P(X>s+t)}{P(X>t)}         \\
	       & = \frac{P(X>s+t\cap X>t)}{P(X>t)} \\
	       & = P(X>s+t\mid X>t)
\end{align*}


\subsection{Risultati notevoli sui modelli}
\begin{prop}
	Siano $X_1,\dots,X_n$ variabili aleatorie indipendenti e sia $Y$ il massimo degli $X_i$, ossia $Y:=\max_i X_i$. Allora:
	\begin{equation*}
		F_Y(x) = \prod_{i=1}^n F_{X_i}(x)
	\end{equation*}
	E per variabili indipendenti e identicamente distribuite (i.i.d.) secondo una funzione di ripartizione $F$:
	\begin{equation*}
		F_Y(x) = \prod_{i=1}^n F(x) = F(x)^n
	\end{equation*}
\end{prop}
\begin{proof}
	\begin{equation*}
		F_Y(x) = P(Y\leq x) = P(\max_i X_i\leq x) = P(\forall i X_i\leq x)
	\end{equation*}
	Dal momento che gli $X_i$ sono indipendenti, l'ultima probabilità è uguale al prodotto delle singole:
	\begin{equation*}
		= \prod_{i=1}^n P(X_i \leq x) = \prod_{i=1}^n F_{X_i}(x)
	\end{equation*}
	Nell'ulteriore ipotesi di variabili indipendenti e identicamente distribuite (i.i.d.) secondo una funzione di ripartizione $F$:
	\begin{equation*}
		= \prod_{i=1}^n F(x) = F(x)^n
	\end{equation*}
\end{proof}

\begin{prop} \label{prop:modnotmin}
	Siano $X_1,\dots,X_n$ variabili aleatorie indipendenti e sia $Z$ il minimo degli $X_i$, ossia $Z:=\min_i X_i$. Allora:
	\begin{equation*}
		F_Z(x) = 1 - \prod_{i=1}^n (1-F_{X_i}(x))
	\end{equation*}
	E per variabili indipendenti e identicamente distribuite (i.i.d.) secondo una funzione di ripartizione $F$:
	\begin{equation*}
		F_Z(x) = 1 - (1-F(x))^n
	\end{equation*}
\end{prop}
\begin{proof}
	\begin{equation*}
		F_Z(x) = 1 - P(Z>x) = 1 - P(\min X_i > x) = 1 - P(\forall i X_i > x)
	\end{equation*}
	Dal momento che gli $X_i$ sono indipendenti, l'ultima probabilità è uguale al prodotto delle singole:
	\begin{equation*}
		= 1 - \prod_{i=1}^n P(X_i > x) = 1 - \prod_{i=1}^n (1-F_{X_i}(x))
	\end{equation*}
	Nell'ulteriore ipotesi di variabili indipendenti e identicamente distribuite (i.i.d.) secondo una funzione di ripartizione $F$:
	\begin{equation*}
		= 1 - \prod_{i=1}^n (1-F(x)) = 1 - (1-F(x))^n
	\end{equation*}
\end{proof}

\begin{prop}
	Siano $X_1,\dots,X_n$ variabili aleatorie indipendenti e sia $Z$ il minimo degli $X_i$, ossia $Z:=\min_i X_i$. Se $X_i\sim E(\lambda_i)$ per ogni $i$, allora:
	\begin{equation*}
		Z\sim E\left(\sum_{i=1}^n \lambda_i\right)
	\end{equation*}
\end{prop}
\begin{proof}
	Essendo le variabili esponenziali le loro funzioni di ripartizione sono del tipo:
	\begin{equation*}
		F_{X_i}(x) = 1-e^{-\lambda_i x}
	\end{equation*}
	Per la proprietà \ref{prop:modnotmin}:
	\begin{align*}
		F_Z(x) & = 1 - \prod_{i=1}^n (1-F_{X_i}(x))   \\
		       & = 1 - \prod_{i=1}^n e^{-\lambda_i x} \\
		       & = 1 - e^{\sum_{i=1}^n -\lambda_i x}  \\
		       & = 1 - e^{-x \sum_{i=1}^n \lambda_i}  \\
	\end{align*}
	Chiamato $\lambda = \sum_{i=1}^n \lambda_i$, allora $Z\sim E(\lambda)$:
	\begin{equation*}
		F_Z(x) = 1 - e^{-\lambda x}
	\end{equation*}
\end{proof}

\begin{prop}
	Se $X\sim E(\lambda)$ e $Y:=cX$ con $c\in\R^+$, allora $Y$ è una variabile aleatoria esponenziale di parametro $\frac{\lambda}{c}$.
	\begin{equation*}
		F_Y(x) = 1 - e^{-\frac{\lambda}{c} x}
	\end{equation*}
\end{prop}
\begin{proof}
	\begin{align*}
		F_Y(x) & =                                  \\
		       & = P(Y \leq x)                      \\
		       & = P(xC \leq x)                     \\
		       & = P\left(X \leq \frac{x}{c}\right) \\
		       & = F_X\left(\frac{x}{c}\right)      \\
		       & = 1 - e^{-\frac{\lambda}{c} x}     \\
	\end{align*}
\end{proof}


\subsection{Modello gaussiano}
Una variabile $X$ di modello gaussiano (o normale), è una variabile aleatoria continua definita da due parametri:
\begin{equation*}
	X\sim N(\mu, \sigma)\qquad \mu\in\R,\sigma\in\R^+
\end{equation*}


\subsubsection{Funzione di densità di probabilità}
\begin{equation*}
	f_X(x) = \frac{1}{\sqrt{2\pi}~\sigma}e^{-\dfrac{(x-\mu)^2}{2\sigma^2}}
\end{equation*}

Studiando la funzione di densità si verifica algebricamente la famosa forma \qt{a campana}:
\begin{align*}
	 & \bullet \lim_{x\to\pm\infty} f_X(x) = 0                                                 \\
	 & \bullet f'_x(x) = \frac{1}{\sqrt{2\pi}~\sigma^3}e^{-\frac{(x-\mu)^2}{2\sigma^2}}(\mu-x) \\
	 & \bullet f'_x(x) \geq 0 \Leftrightarrow x\leq\mu                                         \\
	 & \bullet f''_x(x) = \left(\frac{x-\mu}{\sigma}\right)^2 - 1                              \\
	 & \bullet f''_x(x) \geq 0 \Leftrightarrow x\geq \mu+\sigma \lor x\leq \mu-\sigma
\end{align*}
Modificare $\mu$ significa ovviamente traslare la curva parallelamente all'asse delle $x$. Aumentare il valore di $\sigma$ significa diminuire l'ordinata del massimo e, conseguentemente, \qt{allargare la campana} (in quanto l'area totale sottesa deve rimanere invariata). Vale ovviamente il viceversa per una diminuzione.

Si può dimostrare che l'area sottesa alla funzione di densità è $1$:
\begin{equation*}
	\int_{-\infty}^{+\infty} \frac{1}{\sqrt{2\pi}~\sigma} e^{-\frac{1}{2}\left(\frac{x-\mu}{\sigma}\right)^2} dx = 1
\end{equation*}


\subsubsection{Funzione di ripartizione}
\begin{equation*}
	F_X(x) = \int_{-\infty}^x \frac{1}{\sqrt{2\pi}~\sigma} e^{-\frac{1}{2}\left(\frac{y-\mu}{\sigma}\right)^2} dy
\end{equation*}


\subsubsection{Valore atteso}
\begin{equation*}
	\ev{X} = \mu
\end{equation*}


\subsubsection{Varianza}
\begin{equation*}
	\var(X) = \sigma^2
\end{equation*}


\subsubsection{Distribuzione normale standard}
A partire da una variabile gaussiana $X$ si può costruire la variabile $Z$ come standardizzazione (normalizzazione) di $X$:
\begin{equation*}
	Z = \frac{x-\mu}{\sigma}
\end{equation*}

\noindent
Si verificano i seguenti risultati
\begin{align*}
	\ev{Z}  & = \ev{\frac{1}{\sigma}\ev{X-\mu}}  \\
	        & = \frac{1}{\sigma}(\ev{X}-\mu) = 0 \\
	\\
	\var(Z) & = \frac{1}{\sigma^2} \var(X-\mu)   \\
	        & = \frac{1}{\sigma^2}\var(X) = 1
\end{align*}
Ergo
\begin{equation*}
	X\sim N(\mu,\sigma) \Rightarrow Z\sim N(0,1)
\end{equation*}

Le variabili normali standard si indicano solitamente con $Z$, mentre le relative funzioni di densità e di ripartizione di indicano rispettivamente con $\phi(z)$ e $\Phi(z)$.


\subsubsection{Risultati notevoli}

\paragraph{Trasformazioni lineari} Trasformando linearmente la variabile aleatoria $X\sim N(\mu,\sigma)$, si ottiene una variabile aleatoria gaussiana $Y$:
\begin{equation*}
	X\sim N(\mu,\sigma) \Rightarrow Y\sim N(a\mu+b,a\sigma)
\end{equation*}

\paragraph{Riproducibilità} Date variabili $X_1,\dots,X_n$ gaussiane indipendenti tali che $\forall i X_i\sim N(\mu_i,\sigma_i)$:
\begin{equation*}
	Y\sim N\left(\sum_{i=1}^n \mu_i,\sqrt{\sum_{i=1}^n \sigma_i^2}\right)
\end{equation*}
Anche il modello binomiale e l'ipergeometrico, ad esempio, godono della proprietà di riproducibilità.

\paragraph{Funzione di ripartizione normale e standard} è possibile ricavare la funzione di ripartizione di una variabile aleatoria gaussiana qualsiasi conoscendo la funzione di ripartizione di una variabile standard:
\begin{align*}
	F_X(x) & = P(X\leq x)                                                 \\
	       & = P\left(\frac{X-\mu}{\sigma}\leq\frac{x-\mu}{\sigma}\right) \\
	       & = P\left(Z\leq \frac{x-\mu}{\sigma}\right)                   \\
	       & = F_Z\left(\frac{x-\mu}{\sigma}\right)                       \\
	       & = \Phi\left(\frac{x-\mu}{\sigma}\right)
\end{align*}


\subsection{Risultati notevoli sui modelli}

\subsection{Indici di variabili aleatorie}
\begin{defin}
	Data una variabile aleatoria $X$, la mediana di $X$ è un numero $m\in\R$ tale che $P(X\leq m) = P(X>m) = 1/2$.
\end{defin}

\begin{defin}
	Data una variabile aleatoria $X$, la moda di $X$ è la specificazione di densità (o massa di probabilità) massima.
\end{defin}

% TODO: questa definizione va sistemata: una specificazione come quella descritta non è unica: come ci si comporta?
\begin{defin}
	Data una variabile aleatoria $X$, il quantile di livello $q\in[0,1]$ di $X$ è la specificazione $x_q\in\R$ tale che $P(X\leq x_q) = q$.
\end{defin}


\subsection{Teorema centrale del limite}
\begin{teor}
	Siano $X_1,\dots,X_n$ variabili aleatorie indipendenti e identicamente distribuite, ossia tali che $\forall i\quad\ev{X_i}=\mu\land\var(X_i)=\sigma^2$. Allora per $n$ grandi le variabili sono distribuite in modo approssimativamente\footnote{Il simbolo $\modsim$ indica l'appartenenza approssimativa a un modello.} normale:
	\begin{gather*}
		\sum_{i=1}^n X_i \modsim N(n\mu,\sqrt{n}\sigma) \\
	\end{gather*}
	O, standardizzando
	\begin{equation*}
		\frac{\sum_{i=1}^n X_i-n\mu}{\sqrt{n}\sigma} \modsim N(0,1)
	\end{equation*}
	Ovverosia:
	\begin{equation*}
		\lim_{n\to+\infty} P\left(\frac{\sum_{i=1}^n X_i-n\mu}{\sqrt{n}\sigma}\leq x\right) = \Phi(x)
	\end{equation*}
\end{teor}

\subsubsection{Funzione cumulativa empirica}
La funzione di ripartizione empirica è uno stimatore corretto e consistente della funzione di ripartizione:
\begin{equation*}
	\hat F(x) = \frac{1}{n} \sum_{i=1}^n I_{(-\infty,x]}(x_i)
\end{equation*}
% TODO: check
Fatta una selezione di osservazioni sul campione, a patto che tale selezione sia coerente con la funzione di densità/massa, la funzione cumulativa empirica è un'approssimazione tanto più buona della funzione di ripartizione della selezione quanto grande è la selezione sul campione.

Per il teorema centrale del limite, variabili aleatorie bernoulliane di parametri alti possono essere approssimate con il modello normale:
\begin{gather*}
	X\sim B(n,p) \\
	X = \sum_{i=1}^n X_i \modsim N(np, \sqrt{np(1-p)}) \\
	\frac{x-np}{\sqrt{np(1-p)})} \modsim N(0,1)
\end{gather*}
